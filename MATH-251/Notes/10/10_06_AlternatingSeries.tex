% \documentclass[compacto,10pt]{aleph-notas}
\documentclass[compacto,10pt,comentarios]{aleph-notas}

% -- Paquetes adicionales
\usepackage{enumitem}
\usepackage{aleph-comandos}
\usepackage{parskip}
\usepackage{graphicx}
\usepackage{xfrac}
\usepackage{tikz}
\usepackage{etoolbox}
\usepackage[framemethod=tikz]{mdframed}
\DeclareFontFamily{U}{skulls}{}
\DeclareFontShape{U}{skulls}{m}{n}{ <-> skull }{}
\newcommand{\skull}{\text{\usefont{U}{skulls}{m}{n}\symbol{'101}}}
\def \ds{\displaystyle}
\def \dfx{\dfrac{d}{dx}}
\DeclareMathOperator{\arccot}{arccot}
\DeclareMathOperator{\arcsec}{arcsec}
\DeclareMathOperator{\arccsc}{arccsc}
\newcommand*\Heq{\ensuremath{\overset{\kern2pt LH}{=}}}

% -- Datos del libro
\institucion{Southwestern College}
\asignatura{MATH 251: Calculus II}
\tema{Alternating Series}
\autor{Jesús Pérez Cuarenta}
% \fecha{Fall 2024}

%% --> Logos de las guias
\logouno[4.5cm]{../Images/swc_logo}
\definecolor{colordef}{cmyk}{0.81, 0.62, 0.00, 0.22}

%% -- Solucion para alumnos
% \AtBeginEnvironment{proof}{\color{white}}

\begin{document}

\encabezado
\section*{Alternating Series}
\begin{mdframed}
    \center Learning Objectives \\
    \begin{itemize}
        \item Use the Alternating Series Test (AST) to determine whether an alternating series converges or diverges.
        \item Use the Alternating Series Remainder Theorem to approximate the sum of an alternating series.
    \end{itemize}
\end{mdframed}

\begin{teo}[Alternating Series Test]
    The alternating series
    $$
        \sum_{k=1}^{\infty} (-1)^{k+1} a_k
    $$
    converges provided
    \begin{enumerate}
        \item the terms of the series are nonincreasing in magnitude, i.e.,
        $$
            0 < a_{k+1} \leq a_{k}
        $$
        for $k \in \mathbb{N}$ and,
        \item the limit 
        $$
            \lim_{k \to \infty} a_k = 0
        $$
        is satisfied.
    \end{enumerate}
\end{teo}

\begin{ejer}
    Determine whether the series converges or diverges.
    $$
        \sum_{k=1}^{\infty} (-1)^{k+1} \frac{1}{k}
    $$
\end{ejer}
\begin{proof}[Solution]
    Let $k \in \mathbb{N}$. We have that
    \begin{align*}
    1 & > 0 \\
    \implies k + 1 & > k \\
    \implies \frac{1}{k+1} & < \frac{1}{k}
    \end{align*}
    shows that the terms of the series are nonincreasing, specifically, they are decreasing. Since
    $$
        \lim_{k \to \infty} a_k = \lim_{k \to \infty} \frac{1}{k} = 0
    $$
    we conclude that the alternating series
    $$
        \sum_{k=1}^{\infty} (-1)^{k+1} \frac{1}{k}
    $$
    is convergent.
\end{proof}

\begin{ejer}
    Determine whether the series converges or diverges.
    $$
        \sum_{k=1}^{\infty} (-1)^{k+1} \frac{k^2}{k^2 + 5}
    $$
\end{ejer}
\begin{proof}[Solution]
    We first evaluate the limit and observe
    $$
        \lim_{k \to \infty} \frac{k^2}{k^2 + 5} = 1
    $$
    which means the alternating series test is inconclusive.
\end{proof}

\begin{ejer}
    Determine whether the series converges or diverges.
    $$
        \sum_{k=2}^{\infty} \frac{\cos(\pi k)}{\sqrt{k}}
    $$
\end{ejer}
\begin{proof}[Solution]
    Let $k \in \{2, 3, 4, \ldots \}$. Observe
    $$
        \cos(\pi k) = (-1)^{k}
    $$
    which means we have two conditions to test via
    $$
        a_k = \frac{1}{\sqrt{k}}.
    $$
    First, we have
    \begin{align*}
        1 & > 0 \\
        \implies k + 1 & > k \\
        \implies \sqrt{k+1} & > \sqrt{k} \\
        \implies \frac{1}{\sqrt{k+1}} & < \frac{1}{\sqrt{k}} \\
        \implies a_{k+1} & < a_{k}.
    \end{align*}
    Secondly, the limit yields
    $$
        \lim_{k \to \infty} \frac{1}{\sqrt{k}} = 0.
    $$
    Hence, 
    $$
        \sum_{k=2}^{\infty} \frac{\cos(\pi k)}{\sqrt{k}}
    $$
    is convergent (by the AST).
\end{proof}

\begin{defi}[Absolute and Conditional Convergence]
    Let
    $$
        S = \sum_{k=1}^{\infty} (-1)^{k + 1} a_k
    $$
    be a convergent alternating series.
    \begin{itemize}
        \item The series $S$ is called \textbf{absolutely convergent} if
        $$
            \sum_{k=1}^{\infty} |(-1)^{k + 1} a_k| = \sum_{k=1}^{\infty} |a_k|
        $$
        is convergent.
        \item The series $S$ is called \textbf{conditionally convergent} if
        $$
            \sum_{k=1}^{\infty} |(-1)^{k + 1} a_k| = \sum_{k=1}^{\infty} |a_k|
        $$
        is divergent.
    \end{itemize}
\end{defi}

\begin{ejer}
    Determine whether the series converges absolutely or conditionally.
    $$
        \sum_{k=1}^{\infty} (-1)^{k+1} \frac{1}{k^2 + 8}
    $$
\end{ejer}
\begin{proof}[Solution]
    Let $k \in \mathbb{N}$. First we test for absolute convergence. Consider
    $$
        \sum_{k=1}^{\infty} a_k
        = \sum_{k=1}^{\infty} \frac{1}{k^{2} + 8}
    $$
    and
    $$
        \sum_{k=1}^{\infty} b_k
        = \sum_{k=1}^{\infty} \frac{1}{k^{2}}.
    $$
    We invoke the Limit Comparison Test,
    \begin{align*}
        \lim_{k \to \infty} \frac{a_k}{b_k}
        & = \lim_{k \to \infty} \frac{\frac{1}{k^{2} + 8}}{\frac{1}{k^{2}}} \\
        & = 1.
    \end{align*}
    We get that 
    $$
        \sum_{k=1}^{\infty} \frac{1}{k^{2} + 8}
    $$
    is convergent by the LCT which implies that the infinite series
    $$
        \sum_{k=1}^{\infty} (-1)^{k+1} \frac{1}{k^2 + 8}
    $$
    is \textbf{absolutely convergent}.
\end{proof}

\begin{ejer}
    Determine whether the series converges absolutely or conditionally.
    $$
        \sum_{k=1}^{\infty} (-1)^{k+1} \frac{k+3}{k^2 + 2k + 1}
    $$
\end{ejer}
\begin{proof}[Solution]
    Let $k \in \mathbb{N}$. We first test for absolute convergence. Consider
    $$
        \sum_{k=1}^{\infty} a_k = \sum_{k=1}^{\infty} \frac{k+3}{k^2 + 2k +1}
    $$
    and 
    $$
        \sum_{k=1}^{\infty} b_k = \sum_{k=1}^{\infty} \frac{1}{k}.
    $$
    Now,
    \begin{align*}
        \lim_{k \to \infty} \frac{a_k}{b_k}
        & = \lim_{k \to \infty} \frac{\frac{k+3}{k^2 + 2k +1}}{\frac{1}{k}} \\
        & = \lim_{k \to \infty} \frac{k(k+3)}{k^{2} + 2k + 1} \\
        & = \lim_{k \to \infty} \frac{k^2 + 3k}{k^{2} + 2k + 1} \\ 
        & = 1.
    \end{align*}
    By the Limit Comparison Test, we get that
    $$
        \sum_{k=1}^{\infty} \frac{k+3}{k^2 + 2k +1}
    $$
    is divergent. Now we test for conditional convergence. Let 
    $$
        f(x) = \frac{x+3}{x^2+2x+1}
    $$
    for $x \in \mathbb{R}$. Then,
    $$
        f'(x) = -\frac{x+5}{(x+1)^3} \implies f'(1) < 0.
    $$
    Thus $\{a_k\}$ is a decreasing sequence which implies
    $$
        0 < a_{k+1} < a_{k}.
    $$
    Next, we see that
    $$
        \lim_{k \to \infty} \frac{k+3}{k^2 + 2k + 1} = 0.
    $$
    Therefore,
    $$
        \sum_{k=1}^{\infty} (-1)^{k+1} \frac{k+3}{k^2 + 2k + 1}
    $$
    \textbf{converges conditionally} by the AST.
\end{proof}

\begin{ejer}
    Determine whether the series converges absolutely or conditionally.
    $$
        \sum_{k=1}^{\infty} (-1)^{k} \frac{\ln(k)}{k}
    $$
\end{ejer}
\begin{proof}[Solution]
    Let $k \in \mathbb{N}$. We first test for absolute convergence. Let
    $$
        \sum_{k=1}^{\infty} a_k = \sum_{k=1}^{\infty} \frac{\ln(k)}{k}.
    $$
    Now, note that $f(x) = \ln(x)$ is a strictly increasing function on the interval $(0, \infty)$ (as made evident by $f'(x)$). Then, for $k \geq 3$,
    \begin{align*}
        \ln(k) & > 1 \\
        \implies \frac{\ln(k)}{k} & > \frac{1}{k}.
    \end{align*}
    Since
    $$
        \sum_{k=3}^{\infty} \frac{1}{k}
    $$
    is divergent (harmonic series), by the Direct Comparison Test, we get that
    $$
        \sum_{k=1}^{\infty} a_k = \sum_{k=1}^{\infty} \frac{\ln(k)}{k}
    $$
    is divergent. Now we test for conditional convergence. For $x \in \mathbb{R}$, we have
    \begin{align*}
        g(x) & = \frac{\ln(x)}{x} \\
        \implies g'(x) & = \frac{1-\ln(x)}{x^2} \\
        \implies g'(e^2) & = -\frac{1}{e^{4}} < 0.
    \end{align*}
    Thus, for $k \geq 3$, $a_k$ is decreasing Namely,
    $$
        0 < a_{k+1} \leq a_{k}.
    $$
    In addition,
    $$
        \lim_{k \to \infty} a_k
        = \lim_{k \to \infty} \frac{\ln(k)}{k} 
        = 0.
    $$
    Therefore, the infinite series
    $$
        \sum_{k=1}^{\infty} (-1)^{k} \frac{\ln(k)}{k}
    $$
    \textbf{converges conditionally} by the AST.
\end{proof}

\begin{ejer}
    Determine whether the series converges absolutely or conditionally.
    $$
        \sum_{k=1}^{\infty} (-1)^{k+1} \frac{1}{2^{k}}
    $$
\end{ejer}
\begin{proof}
    Let $k \in \mathbb{N}$ and notice that
    $$
        \sum_{k=1}^{\infty} a_k = \sum_{k=1}^{\infty} \frac{1}{2^{k}}
    $$
    is a convergent series (geometric series). Therefore, the infinite series
    $$
        \sum_{k=1}^{\infty} (-1)^{k+1} \frac{1}{2^{k}}
    $$
    \textbf{converges absolutely}.
\end{proof}

\begin{teo}[Remainder in Alternating Series]
    Let
    $$
        \sum_{k=1}^{\infty} (-1)^{k+1} a_k
    $$
    be a convergent alternating series with terms that are nonincreasing in magnitude. Let
    $$
        R_n = S - S_n
    $$
    be the remainder in approximating the value of that series by the sum of its first $n$ terms.
    Then
    $$
        |R_n| \leq a_{n+1}.
    $$
\end{teo}

\begin{ejer}
    Given
    $$
        \sum_{k=1}^{\infty} (-1)^{k+1} \frac{1}{k} = \ln(2).
    $$
    How many terms of the series are required to approximate $\ln(2)$ with an error less than $10^{-6}$?
\end{ejer}
\begin{proof}[Solution]
    Let $k, n \in \mathbb{N}, n > k$, and
    $$
        S = S_{n} + R_{n}
    $$
    where
    $$
        S_{n} = \sum_{k=1}^{n} (-1)^{k+1} \frac{1}{k}
    $$
    and
    $$
        R_{n} = \sum_{k=n+1}^{\infty} (-1)^{k+1} \frac{1}{k}.
    $$
    We have
    \begin{align*}
        |R_{n}| & \leq \left|(-1)^{(n+2)} \frac{1}{n + 1} \right| \\
        \implies |R_{n}| & \leq \frac{1}{n+1}.
    \end{align*}
    We now seek a value of $n$ satisfying the following inequalities,
    \begin{align*}
        \frac{1}{n+1} & \leq 10^{-6} \\
        \implies 10^{6} & \leq n + 1 \\
        \implies 10^{6} - 1 & \leq n.
    \end{align*}
    With numerical methods, one can show
    $$
        \left| \ln(2) - \sum_{k=1}^{10^{6} - 1} (-1)^{k+1} \frac{1}{k} \right| \approx 5 \cdot 10^{-7} < 10^{-6}
    $$
    confirming our choice of $n \geq 999999$.
\end{proof}
\end{document}