% \documentclass[compacto,10pt]{aleph-notas}
\documentclass[compacto,10pt,comentarios]{aleph-notas}

% -- Paquetes adicionales
\usepackage{enumitem}
\usepackage{aleph-comandos}
\usepackage{parskip}
\usepackage{graphicx}
\usepackage{xfrac}
\usepackage{tikz}
\usepackage{etoolbox}
\usepackage[framemethod=tikz]{mdframed}
\DeclareFontFamily{U}{skulls}{}
\DeclareFontShape{U}{skulls}{m}{n}{ <-> skull }{}
\newcommand{\skull}{\text{\usefont{U}{skulls}{m}{n}\symbol{'101}}}
\def \ds{\displaystyle}
\def \dfx{\dfrac{d}{dx}}
\DeclareMathOperator{\arccot}{arccot}
\DeclareMathOperator{\arcsec}{arcsec}
\DeclareMathOperator{\arccsc}{arccsc}
\newcommand*\Heq{\ensuremath{\overset{\kern2pt LH}{=}}}

% -- Datos del libro
\institucion{Southwestern College}
\asignatura{MATH 251: Calculus II}
\tema{Alternating Series}
\autor{Jesús Pérez Cuarenta}
% \fecha{Fall 2024}

%% --> Logos de las guias
\logouno[4.5cm]{../Images/swc_logo}
\definecolor{colordef}{cmyk}{0.81, 0.62, 0.00, 0.22}

%% -- Solucion para alumnos
% \AtBeginEnvironment{proof}{\color{white}}

\begin{document}

\encabezado
\section*{Alternating Series}
\begin{mdframed}
    \center Learning Objectives \\
    \begin{itemize}
        \item Use the Alternating Series Test to determine whether an alternating series converges (absolutely or conditionally) or diverges.
        \item Use the Alternating Series Remainder Theorem to approximate the sum of an alternating series.
    \end{itemize}
\end{mdframed}

\begin{teo}[Alternating Series Test]
    The alternating series
    $$
        \sum_{k=1}^{\infty} (-1)^{k+1} a_k
    $$
    converges provided
    \begin{enumerate}
        \item the terms of the series are nonincreasing in magnitude, i.e.,
        $$
            0 < a_{k+1} \leq a_{k}
        $$
        for $k$ greater than some $N \in \mathbb{N}$ and,
        \item the limit 
        $$
            \lim_{k \to \infty} a_k = 0
        $$
        is satisfied.
    \end{enumerate}
\end{teo}

\begin{ejer}
    Determine whether the series converges or diverges.
    $$
        \sum_{k=1}^{\infty} (-1)^{k+1} \frac{1}{k}
    $$
\end{ejer}
\begin{proof}[Solution]
    Let $k \in \mathbb{N}$. We have that
    \begin{align*}
    1 & > 0 \\
    \implies k + 1 & > k \\
    \implies \frac{1}{k+1} & < \frac{1}{k}
    \end{align*}
    shows that the terms of the series are nonincreasing, specifically, they are decreasing. Since
    $$
        \lim_{k \to \infty} a_k = \lim_{k \to \infty} \frac{1}{k} = 0
    $$
    we conclude that the alternating series
    $$
        \sum_{k=1}^{\infty} (-1)^{k+1} \frac{1}{k}
    $$
    is convergent.
\end{proof}

\begin{ejer}
    Determine whether the series converges or diverges.
    $$
        \sum_{k=1}^{\infty} (-1)^{k+1} \frac{k^2}{k^2 + 5}
    $$
\end{ejer}
\begin{proof}[Solution]
    We first evaluate the limit
    $$
        \lim_{k \to \infty} \frac{k^2}{k^2 + 5} = 1
    $$
    which means the alternating series test is inconclusive.
\end{proof}

\begin{ejer}
    Determine whether the series converges or diverges.
    $$
        \sum_{k=2}^{\infty} \frac{\cos(\pi k)}{\sqrt{k}}
    $$
\end{ejer}
\begin{proof}[Solution]
    Let $k \in \{2, 3, 4, \ldots \}$. Observe
    $$
        \cos(\pi k) = (-1)^{k}
    $$
    which means we have two conditions to test via
    $$
        a_k = \frac{1}{\sqrt{k}}.
    $$
    First, we have
    \begin{align*}
        1 & > 0 \\
        \implies k + 1 & > k \\
        \implies \sqrt{k+1} & > \sqrt{k} \\
        \implies \frac{1}{\sqrt{k+1}} & < \frac{1}{\sqrt{k}} \\
        \implies a_{k+1} & < a_{k}.
    \end{align*}
    Secondly, the limit yields
    $$
        \lim_{k \to \infty} \frac{1}{\sqrt{k}} = 0.
    $$
    Hence, 
    $$
        \sum_{k=2}^{\infty} \frac{\cos(\pi k)}{\sqrt{k}}
    $$
    is convergent.
\end{proof}

\begin{defi}[Absolute and Conditional Convergence]
    Let
    $$
        S = \sum_{k=1}^{\infty} a_k
    $$
    be a convergent alternating series.
    \begin{itemize}
        \item The series $S$ is called \textbf{absolutely convergent} if
        $$
            \sum_{k=1}^{\infty} |a_k|
        $$
        is convergent.
        \item The series $S$ is called \textbf{conditionally convergent} if
        $$
            \sum_{k=1}^{\infty} |a_k|
        $$
        is divergent.
    \end{itemize}
\end{defi}

\begin{ejer}
    Determine whether the series converges absolutely or conditionally.
    $$
        \sum_{k=1}^{\infty} (-1)^{k+1} \frac{1}{k^2 + 8}
    $$
\end{ejer}

\begin{ejer}
    Determine whether the series converges absolutely or conditionally.
    $$
        \sum_{k=1}^{\infty} (-1)^{k+1} \frac{k+3}{k^2 + 2k + 1}
    $$
\end{ejer}

\begin{ejer}
    Determine whether the series converges absolutely or conditionally.
    $$
        \sum_{k=1}^{\infty} (-1)^{k} \frac{\ln(k)}{k}
    $$
\end{ejer}

\begin{ejer}
    Determine whether the series converges absolutely or conditionally.
    $$
        \sum_{k=1}^{\infty} (-1)^{k+1} \frac{1}{2^{k}}
    $$
\end{ejer}

\begin{ejer}
    Determine whether the series converges absolutely or conditionally.
    $$
        \sum_{k=1}^{\infty} \frac{\sin(k)}{k^3}
    $$
\end{ejer}

\begin{teo}[Remainder in Alternating Series]
    Let
    $$
        \sum_{k=1}^{\infty} (-1)^{k+1} a_k
    $$
    be a convergent alternating series with terms that are nonincreasing in magnitude. Let
    $$
        R_n = S - S_n
    $$
    be the remainder in approximating the value of that series by the sum of its first $n$ terms.
    Then
    $$
        |R_n| \leq a_{n+1}.
    $$
\end{teo}

\begin{ejer}
    Given
    $$
        \sum_{k=1}^{\infty} (-1)^{k+1} \frac{1}{k} = \ln(2).
    $$
    How many terms of the series are required to approximate $\ln(2)$ with an error less than $10^{-6}$?
\end{ejer}
\end{document}