% \documentclass[compacto,10pt]{aleph-notas}
\documentclass[compacto,10pt,comentarios]{aleph-notas}

% -- Paquetes adicionales
\usepackage{enumitem}
\usepackage{aleph-comandos}
\usepackage{parskip}
\usepackage{graphicx}
\usepackage{xfrac}
\usepackage{tikz}
\usepackage{etoolbox}
\usepackage[framemethod=tikz]{mdframed}
\DeclareFontFamily{U}{skulls}{}
\DeclareFontShape{U}{skulls}{m}{n}{ <-> skull }{}
\newcommand{\skull}{\text{\usefont{U}{skulls}{m}{n}\symbol{'101}}}
\def \ds{\displaystyle}
\def \dfx{\dfrac{d}{dx}}
\DeclareMathOperator{\arccot}{arccot}
\DeclareMathOperator{\arcsec}{arcsec}
\DeclareMathOperator{\arccsc}{arccsc}
\newcommand*\Heq{\ensuremath{\overset{\kern2pt LH}{=}}}

% -- Datos del libro
\institucion{Southwestern College}
\asignatura{MATH 251: Calculus II}
\tema{Infinite Series}
\autor{Jesús Pérez Cuarenta}
% \fecha{Fall 2024}

%% --> Logos de las guias
\logouno[4.5cm]{../Images/swc_logo}
\definecolor{colordef}{cmyk}{0.81, 0.62, 0.00, 0.22}

%% -- Solucion para alumnos
% \AtBeginEnvironment{proof}{\color{white}}

\begin{document}

\encabezado

\section*{Infinite Series}
\begin{mdframed}
    \center Learning Objectives \\
    \begin{itemize}
        \item Explain the meaning of the sum of an infinite series.
        \item Calculate the sum of a geometric series.
        \item Evaluate a telescoping series.
    \end{itemize}
\end{mdframed}

An infinite series is a sum of infinitely many terms and is written in the form
$$
    \sum_{n = 1}^{\infty} a_n = a_1 + a_2 + a_3 + \ldots.
$$
\textbf{What does it mean to add infinitely many terms?}

\begin{ejer}
    $\skull$ Find a value tied to the infinite sum
    $$
        S_{a} = 1 - 1 + 1 - 1 + \ldots.
    $$
\end{ejer}
\begin{proof}[Solution]
    Consider the sequence
    $$
        \{ a_n \}_{n=1}^{\infty} = \{(-1)^{n+1}\}_{n=1}^{\infty} = \{1, -1, 1, -1, 1, \ldots \}.
    $$
    We have
    \begin{align*}
        S_{a} & = a_1 + a_2 + a_3 + a_4 + \ldots \\
        \implies S_{a} & = 1 - 1 + 1 - 1 + \ldots \\
        \implies S_{a} & = 1 - (1 + 1 - 1 + \ldots) \\
        \implies S_{a} & = 1 - S_{a} \\
        \implies 2S_{a} & = 1 \\
        \implies S_{a} & = \frac{1}{2}.
    \end{align*}
    Hence,
    $$
        1 - 1 + 1 - 1 + \ldots = \frac{1}{2}.
    $$
\end{proof}

\begin{ejer}
    $\skull$ Find a value tied to the infinite sum
    $$
        S_{b} = 1 - 2 + 3 -4 + \ldots.
    $$
    Use the value obtained for $S_a$ in Example 1.
\end{ejer}
\begin{proof}[Solution]
    Take the sequence
    $$
        \{ b_n \}_{n=1}^{\infty} = \{(-1)^{n+1} n\}_{n=1}^{\infty} = \{1, -2, 3, -4, 5, \ldots \}.
    $$
    Notice that
    \begin{align*}
        S_{b} & = b_1 + b_2 + b_3 + b_4 \ldots \\
        S_{b} & = 1 -2 + 3 - 4 \ldots \\
        S_{b} & = 0 + 1 - 2 + 3 \ldots \\
        \implies 2S_{b} & = 1 -1 + 1 - 1 \ldots \\ 
        \implies 2S_{b} & = S_{a} \\
        \implies S_{b} & = \frac{1}{2}S_{a} \\
        \implies S_{b} & = \frac{1}{4}.
    \end{align*}
    Thus,
    $$
        1 -2 + 3 - 4 + \ldots = \frac{1}{4}.
    $$
\end{proof}

\begin{ejer}
    $\skull$ Find a value tied to the infinite sum
    $$
        S_{c} = 1 + 2 + 3 + 4 + \ldots.
    $$
    Use the value obtained for $S_b$ in Example 2.
\end{ejer}
\begin{proof}[Solution]
    Define
    $$
        \{c_n\}_{n=1}^{\infty} = \{n\}_{n=1}^{\infty} = \{1, 2, 3, 4, \ldots\}.
    $$
    Now,
    \begin{align*}
        S_{c} & = c_1 + c_2 + c_3 + c_4 + \ldots \\
        \implies S_{c} & = 1 + 2 + 3 + 4 + \ldots.
    \end{align*}
    On the other hand,
    \begin{align*}
        S_{b} & = b_1 + b_2 + b_3 + b_4 \ldots \\
        \implies S_{b} & = 1 - 2 + 3 - 4 + \ldots.
    \end{align*}
    Now we compute $S_{c} - S_{b}$,
    \begin{align*}
        S_{c} - S_{b} & =  (1 - 1) + (2 + 2) + (3 - 3) + (4 + 4) + \ldots \\
        \implies S_{c} - S_{b} & = 4(1 + 2 + 3 + 4 + \ldots) \\
        \implies S_{c} - S_{b} & = 4S_{c} \\
        \implies 3S_{c} & = -\frac{1}{4} \\
        \implies S_{c} & = -\frac{1}{12} .
    \end{align*}
    Therefore,
    $$
        1 + 2 + 3 + 4 + \ldots = -\frac{1}{12}.
    $$
\end{proof}

The moral of the story is we need to be careful when adding terms from an infinite sequence.
\begin{defi}[Infinite Series]
    Given a sequence $\{a_1, a_2, a_3, \ldots \}$, the sum of its terms
    $$
        a_1+a_2+a_3+\ldots = \sum_{k=1}^{\infty} a_k
    $$
    is called an \textbf{infinite series}. The \textbf{sequence of partial sums} $\{S_n\}$ associated with this series has the terms
    \begin{align*}
        S_1 & = a_1 \\
        S_2 & = a_1 + a _2 \\
        S_3 & = a_1 + a_2 + a_3 \\
        & \vdots \\
        S_n & = a_1 + a_2 + a_3 + \ldots + a_n = \sum_{k=1}^{n} a_k
    \end{align*}
    for $n \in \mathbb{N}$. If the sequence of partial sums $\{S_n\}$ has a limit $L$, the infinite series \textbf{converges} to that limit, and we write
    $$
        \sum_{k=1}^{\infty} a_k = \lim_{n \to \infty} \sum_{k=1}^{n} a_k = \lim_{n \to \infty} S_n = L.
    $$
    If the sequence of partial sums diverges, the infinite series also \textbf{diverges}.
\end{defi}
\begin{ejer}
    Show that the infinite series 
    $$
        \sum_{k=1}^\infty \frac{1}{2^k}
    $$ converges.
\end{ejer}
\begin{proof}[Solution]
    We have
    \begin{align*}
        S_1 & = \frac{1}{2} \\
        S_2 & = \frac{1}{2} + \frac{1}{4} = \frac{3}{4} \\
        S_3 & = \frac{1}{2} + \frac{1}{4} + \frac{1}{8}  = \frac{7}{8} \\
        S_4 & = \frac{1}{2} + \frac{1}{4} + \frac{1}{8} + \frac{1}{16} = \frac{15}{16} \\
        & \vdots \\
        S_n & = \frac{2^{n} - 1}{2^{n}} = 1 - \frac{1}{2^{n}}.
    \end{align*}
    Taking the limit, we get
    $$
        \lim_{n \to \infty} S_n = \lim_{n \to \infty} 1 - \frac{1}{2^{n}} = 1.
    $$
    Therefore, the infinite series converges and
    $$
        \sum_{k=1}^\infty \frac{1}{2^k} = 1.
    $$
\end{proof}

\begin{defi}[Geometric Series]
    A \textbf{geometric series} with ratio $r \in \mathbb{R}$ and $a \neq 0$ is a series of the form
    $$
        \sum_{k=0}^{n-1}ar^k = a (1 + r + r^2 + r^3 + \ldots + r^{n-1}).
    $$
    An \textbf{infinite geometric series} is of the form
    $$
        \sum_{k=0}^{\infty}ar^k = a (1 + r + r^2 + r^3 + \ldots ).
    $$
\end{defi}

\begin{teo}[Convergence of Geometric Series]
    Let $a \neq 0$ and $r$ be real numbers. If $|r| < 1$, then
    $$
        \sum_{k=0}^{\infty} a r^{k} = \frac{a}{1 - r}.
    $$
    If $|r| \geq 1$, then the series diverges.
\end{teo}

\begin{ejer}
    Show that the infinite series
    $$
        \sum_{k=0}^{\infty} \mathrm{e}^{-k}
    $$
    is a geometric series and find its sum if it is convergent.
\end{ejer}
\begin{proof}[Solution]
    We rewrite the problem as
    $$
        \sum_{k=0}^{\infty} \mathrm{e}^{-k} = \sum_{k=0}^{\infty} \left( \frac{1}{e} \right)^{k}.
    $$
    We identify $a = 1$ and $r = e^{-1}$ and note that the series is convergent. In addition,
    \begin{align*}
        \sum_{k=0}^{\infty} \left( \frac{1}{e} \right)^{k}
        & = \frac{1}{1 - \mathrm{e}^{-1}} \\
        & = \frac{\mathrm{e}}{\mathrm{e}-1}.
    \end{align*}
\end{proof}

\begin{ejer}
    Show that the infinite series
    $$
        \sum_{k=0}^{\infty} 3\left(\frac{5}{2}\right)^{k}
    $$
    is a geometric series and find its sum if it is convergent.
\end{ejer}
\begin{proof}[Solution]
    We identify $a = 3$ and $r = \frac{5}{2}$. We see that the series is geometric and is divergent since $|r| \geq 1$.
\end{proof}

\begin{ejer}
    Show that the infinite series
    $$
        \sum_{k=1}^{\infty} 5 \left( - \frac{3}{4} \right)^{k}
    $$
    is a geometric series and find its sum if it is convergent.
\end{ejer}
\begin{proof}[Solution]
    First, we shift the index via $j = k - 1$ and get
    $$
    \sum_{k=1}^{\infty} 5 \left( - \frac{3}{4} \right)^{k}
    = \sum_{j=0}^{\infty} 5 \left( - \frac{3}{4} \right)^{j+1}
    = \sum_{j=0}^{\infty} -\frac{15}{4} \left( - \frac{3}{4} \right)^{j}.
    $$
    We identify $a = -\frac{15}{4}$ and $r = -\frac{3}{4}$ and conclude the series is convergent. Moreover,
    \begin{align*}
    \sum_{j=0}^{\infty} -\frac{15}{4} \left( - \frac{3}{4} \right)^{j}
    & = -\frac{15}{4} \left( \frac{1}{1 + \frac{3}{4}} \right) \\
    & = -\frac{15}{4} \left( \frac{4}{7} \right) \\
    & = \frac{15}{7}.
    \end{align*}
\end{proof}

\begin{ejer}
    Show that the infinite series
    $$
        \sum_{k=3}^{\infty} 9 \left( \frac{3}{7} \right)^{k}
    $$
    is a geometric series and find its sum if it is convergent.
\end{ejer}
\begin{proof}[Solution]
    First, we shift the index via $j = k - 3$ and get
    $$
    \sum_{k=3}^{\infty} 9 \left( \frac{3}{7} \right)^{k}
    = \sum_{j=0}^{\infty} 9 \left( \frac{3}{7} \right)^{j+3}
    = \sum_{j=0}^{\infty} \frac{243}{343} \left( \frac{3}{7} \right)^{j}.
    $$
    We identify $a = \frac{243}{343}$ and $r = \frac{3}{7}$ and conclude the series is convergent. Moreover,
    \begin{align*}
    \sum_{j=0}^{\infty} \frac{243}{343} \left( \frac{3}{7} \right)^{j}
    & = \frac{243}{343} \left( \frac{1}{1 - \frac{3}{7}} \right) \\
    & = \frac{243}{343} \left( \frac{7}{4} \right) \\
    & = \frac{243}{196}.
    \end{align*}
\end{proof}

\begin{ejer}
    Show that the infinite series
    $$
        \sum_{k=1}^{\infty} \frac{1}{k^2+k}
    $$
    converges.
\end{ejer}
\begin{proof}[Solution]
    By partial fraction decomposition we get
    $$
        \frac{1}{k^2 + k} = \frac{1}{k} - \frac{1}{k+1}
    $$
    so that
    $$
        \sum_{k=1}^{\infty} \frac{1}{k^2+k}
        = \sum_{k=1}^{\infty} \left( \frac{1}{k} - \frac{1}{k+1} \right).
    $$
    Define
    $$
        S_n = \sum_{k=1}^{n} \left( \frac{1}{k} - \frac{1}{k+1} \right)
    $$
    and notice
    \begin{align*}
        S_1 & = \left(1 - \frac{1}{2} \right) \\
        S_2 & = \left(1 - \frac{1}{2} \right) +  \left(\frac{1}{2} - \frac{1}{3} \right)\\
        S_3 & = \left(1 - \frac{1}{2} \right) +  \left(\frac{1}{2} - \frac{1}{3} \right)
            + \left(\frac{1}{3} - \frac{1}{4} \right) \\
            & \vdots \\
        S_n & = \left(1 - \frac{1}{2} \right) +  \left(\frac{1}{2} - \frac{1}{3} \right) 
            + \ldots + \left(\frac{1}{n-1} - \frac{1}{n} \right)
            + \left( \frac{1}{n} - \frac{1}{n+1} \right) \\
            & = 1 - \frac{1}{n+1}.
    \end{align*}
    Taking the limit shows us that
    $$
        \lim_{n \to \infty} S_n = \lim_{n \to \infty} 1 - \frac{1}{n+1} = 1.
    $$
    Therefore,
    $$
        \sum_{k=1}^{\infty} \frac{1}{k^2+k} = 1.
    $$    
    and we conclude the series is convergent.
\end{proof}

\begin{ejer}
    Show that the infinite series
    $$
        \sum_{k=1}^{\infty} \left( \frac{1}{2k+1} - \frac{1}{2k+5} \right)
    $$
    converges.
\end{ejer}
\begin{proof}[Solution]
    Define
    $$
        S_n = \sum_{k=1}^{n} \left( \frac{1}{2k+1} - \frac{1}{2k+5} \right)
    $$
    and notice
    \begin{align*}
     S_n & = \sum_{k=1}^{n} \left( \frac{1}{2k+1} - \frac{1}{2k+5} \right) \\
        & = \sum_{k=1}^{n}\frac{1}{2k+1} - \sum_{k=1}^{n} \frac{1}{2k+5} \\
        & = \frac{1}{3} + \frac{1}{5} + \sum_{k=3}^{n}\frac{1}{2k+1} - \sum_{k=1}^{n} \frac{1}{2k+5} \\
        & = \frac{1}{3}
            + \frac{1}{5} 
            + \ldots 
            - \frac{1}{2n+1}
            + \frac{1}{2n+1} - \frac{1}{2n+5} \\
        & = \frac{8}{15} - \frac{1}{2n+5}.
    \end{align*}
    Taking the limit shows us that
    $$
        \lim_{n \to \infty} S_n
        = \lim_{n \to \infty} \left( \frac{8}{15} - \frac{1}{2n+5} \right)
        = \frac{8}{15}.
    $$
    Therefore,
    $$
        \sum_{k=1}^{\infty} \left( \frac{1}{2k+1} - \frac{1}{2k+5} \right) = \frac{8}{15}.
    $$    
    and we conclude the series is convergent.
\end{proof}

\begin{teo}[Properties of Convergent Series]
    \begin{enumerate}
        \item Suppose $\sum a_k$ converges to $A$ and $c$ is a real number. The series $\sum c a_k$ converges, and
        $$
            \sum c a_k = c \sum a_k = c A.
        $$ 
        \item Suppose $\sum a_k$ diverges. Then $\sum c a_k$ also diverges, for any real number $c \neq 0$.
        \item Suppose $\sum a_k$ converges to $A$ and $\sum b_k$ converges to $B$. The series $\sum (a_k \pm b_k)$ converges, and
        $$
            \sum(a_k \pm b_k) = \sum a_k \pm \sum b_k = A \pm B.
        $$
        \item Suppose $\sum a_k$ diverges and $\sum b_k$ converges, Then 
        $$
            \sum (a_k \pm b_k)
        $$
        diverges.
        \item If $M$ is a positive integer, then
        $$
            \sum_{k=1}^{\infty} a_k
        $$
        and
        $$
            \sum_{k=M}^{\infty} a_k
        $$
        either both converge or both diverge. In general, whether a series converges does not depend on a finite number of terms added to or removed from the series. However, the value of a convergent series does change if nonzero terms are added or removed.
    \end{enumerate}
\end{teo}

\end{document}