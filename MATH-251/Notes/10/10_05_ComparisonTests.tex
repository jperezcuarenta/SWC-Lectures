% \documentclass[compacto,10pt]{aleph-notas}
\documentclass[compacto,10pt,comentarios]{aleph-notas}

% -- Paquetes adicionales
\usepackage{enumitem}
\usepackage{aleph-comandos}
\usepackage{parskip}
\usepackage{graphicx}
\usepackage{xfrac}
\usepackage{tikz}
\usepackage{etoolbox}
\usepackage[framemethod=tikz]{mdframed}
\DeclareFontFamily{U}{skulls}{}
\DeclareFontShape{U}{skulls}{m}{n}{ <-> skull }{}
\newcommand{\skull}{\text{\usefont{U}{skulls}{m}{n}\symbol{'101}}}
\def \ds{\displaystyle}
\def \dfx{\dfrac{d}{dx}}
\DeclareMathOperator{\arccot}{arccot}
\DeclareMathOperator{\arcsec}{arcsec}
\DeclareMathOperator{\arccsc}{arccsc}
\newcommand*\Heq{\ensuremath{\overset{\kern2pt LH}{=}}}

% -- Datos del libro
\institucion{Southwestern College}
\asignatura{MATH 251: Calculus II}
\tema{Comparison Tests}
\autor{Jesús Pérez Cuarenta}
% \fecha{Fall 2024}

%% --> Logos de las guias
\logouno[4.5cm]{../Images/swc_logo}
\definecolor{colordef}{cmyk}{0.81, 0.62, 0.00, 0.22}

%% -- Solucion para alumnos
% \AtBeginEnvironment{proof}{\color{white}}

\begin{document}

\encabezado
\section*{Comparison Tests}
\begin{mdframed}
    \center Learning Objectives \\
    \begin{itemize}
        \item Define the Direct Comparison Test (DCT) and the Limit Comparison Test (LCT).
        \item Determine convergence and divergence of infinite series using the DCT and LCT.
    \end{itemize}
\end{mdframed}

\begin{teo}[Direct Comparison Test]
    Let $\sum a_k$ and $\sum b_k$ be two infinite series with $a_k, b_k \geq 0$ for $k \in \mathbb{N}$ and $a_k \leq b_k$. Then,
    \begin{enumerate}
        \item If $\sum b_k$ is convergent then so is $\sum a_k$.
        \item If $\sum a_k$ is divergent then so is $\sum b_k$.
    \end{enumerate}
\end{teo}

\begin{ejer}
    Determine whether the following series converges or diverges.
    $$
        \sum_{k=1}^{\infty} \frac{1}{k^2 + 2}
    $$
\end{ejer}
\begin{proof}[Solution]
    Let $k \in \mathbb{N}$. We have
    \begin{align*}
        2 & > 0 \\
        \implies k^2 + 2 & > k^2 \\
        \implies \frac{1}{k^2 + 2} & < \frac{1}{k^2}.
    \end{align*}
    Thus $\frac{1}{k^2 + 2}$ is bounded from above by $\frac{1}{k^2}$. Since
    $$
        \sum_{k=1}^{\infty} \frac{1}{k^2}
    $$
    converges ($p$-series with $p=2$), we get that
    $$
        \sum_{k=1}^{\infty} \frac{1}{k^2 + 2}
    $$ is convergent.
\end{proof}

\begin{ejer}
    Determine whether the following series converges or diverges.
    $$
        \sum_{k=1}^{\infty} \frac{1}{3 + 2^{k}}
    $$
\end{ejer}
\begin{proof}[Solution]
    Let $k \in \mathbb{N}$. We have
    \begin{align*}
        3 & > 0 \\
        \implies 3 + 2^{k} & > 2^{k} \\
        \implies \frac{1}{3 + 2^{k}} & < \frac{1}{2^{k}}.
    \end{align*}
    Thus $\frac{1}{3 + 2^{k}}$ is bounded from above by $\frac{1}{2^{k}}$. Since
    $$
        \sum_{k=1}^{\infty} \frac{1}{2^{k}}
    $$
    converges (geometric series with $r = \frac{1}{2}$), we get that
    $$
        \sum_{k=1}^{\infty} \frac{1}{3 + 2^{k}}
    $$ is convergent.
\end{proof}

\begin{ejer}
    Determine whether the following series converges or diverges.
    $$
        \sum_{k=1}^{\infty} \frac{1}{\sqrt{k} - 4}
    $$
\end{ejer}
\begin{proof}[Solution]
    Let $k \in \mathbb{N}$. We have
    \begin{align*}
        -4 & < 0 \\
        \implies \sqrt{k} - 4 & < \sqrt{k} \\
        \implies \frac{1}{\sqrt{k} - 4} & > \frac{1}{\sqrt{k}}.
    \end{align*}
    Thus $\frac{1}{\sqrt{k} - 4}$ is bounded from below by $\frac{1}{\sqrt{k}}$. Since
    $$
        \sum_{k=1}^{\infty} \frac{1}{\sqrt{k}}
    $$
    divergent ($p$-series with $p = \frac{1}{2}$), we get that
    $$
        \sum_{k=1}^{\infty} \frac{1}{\sqrt{k} - 4}
    $$ is divergent.
\end{proof}

\begin{ejer}
    $\skull$
    Determine whether the following series converges or diverges.
    $$
        \sum_{k=1}^{\infty} \frac{1}{\sqrt{k} + 4}
    $$
\end{ejer}
\begin{proof}[Solution]
    Let $k \in \mathbb{N}$. We have
    \begin{align*}
        4 & > 0 \\
        \implies \sqrt{k} + 4 & > \sqrt{k} \\
        \implies \frac{1}{\sqrt{k} + 4} & < \frac{1}{\sqrt{k}}.
    \end{align*}
    Thus $\frac{1}{\sqrt{k} + 4}$ is bounded from above by $\frac{1}{\sqrt{k}}$. Hence, we are unable to reach a conclusion via the Direct Comparison Test.
\end{proof}

\begin{teo}[Limit Comparison Test]
    Let $\sum a_k$ and $\sum b_k$ be two infinite series with $a_k \geq 0$ and $b_k > 0$ for $k \in \mathbb{N}$. Define
    $$
        c := \lim_{k \to \infty} \frac{a_k}{b_k}.
    $$
    If $c$ is positive ($c > 0$) and finite ($c < \infty$) then either both series converge or both series diverge.
\end{teo}

\begin{ejer}
    Determine whether the following series converges or diverges.
    $$
        \sum_{k=1}^{\infty} \frac{1}{2^{k} - 1}
    $$
\end{ejer}
\begin{proof}[Solution]
    Let $a_k = \frac{1}{2^{k} - 1}$ and $b_k = \frac{1}{2^{k}}$ for $k \in \mathbb{N}$. Then,
    \begin{align*}
        \lim_{k \to \infty} \frac{a_k}{b_k}
            & = \lim_{k \to \infty} \frac{\frac{1}{2^{k} - 1}}{\frac{1}{2^{k}}} \\
            & = \lim_{k \to \infty} \frac{2^{k}}{2^{k} - 1} \\
            & = \lim_{k \to \infty} \left( \frac{2^{k}}{2^{k} - 1} \right) \left( \frac{2^{-k}}{2^{-k}} \right) \\
            & = \lim_{k \to \infty} \frac{1}{1 - \frac{1}{2^{k}}} \\
            & = 1.
    \end{align*}
    Since 
    $$
    \lim_{k \to \infty} \frac{a_k}{b_k} = 1
    $$
    is positive, finite, and
    $$
        \sum_{k=1}^{\infty} \frac{1}{2^{k}}
    $$
    is convergent (geometric series with $r=\frac{1}{2}$), we conclude that 
    $$
        \sum_{k=1}^{\infty} \frac{1}{2^{k} - 1}
    $$
    is convergent.
\end{proof}

\begin{ejer}
    Determine whether the following series converges or diverges.
    $$
        \sum_{k=1}^{\infty} \frac{7^{k}}{3^{k} + 9k  - 3}
    $$
\end{ejer}
\begin{proof}[Solution]
    Let
    $$
        a_k = \frac{7^{k}}{3^{k} + 9k  - 3}
    $$
    and
    $$
        b_k = \frac{7^{k}}{3^{k}} 
    $$
    for $ k \in \mathbb{N}$. Then,
    \begin{align*}
        \lim_{k \to \infty} \frac{a_k}{b_k}
            & = \lim_{k \to \infty} \frac{\frac{7^{k}}{3^{k} + 9k  - 3}}{\frac{7^{k}}{3^{k}} } \\
            & = \lim_{k \to \infty} \frac{3^{k}7^{k}}{(3^{k} + 9k -3)7^{k}} \\
            & = \lim_{k \to \infty} \frac{3^{k}}{(3^{k} + 9k -3)} \left( \frac{3^{-k}}{3^{-k}} \right) \\
            & = \lim_{k \to \infty} \frac{1}{(1 + \frac{9k}{3^{k}} - \frac{3}{3^{k}})} \\
            & = 1.
    \end{align*}
    Since 
    $$
    \lim_{k \to \infty} \frac{a_k}{b_k} = 1
    $$
    is positive, finite, and
    $$
        \sum_{k=1}^{\infty} \frac{7^{k}}{3^{k}}
    $$
    is divergent (geometric series with $r=\frac{7}{3}$), we conclude that 
    $$
        \sum_{k=1}^{\infty} \frac{7^{k}}{3^{k} + 9k  - 3}
    $$
    is divergent.
\end{proof}

\begin{ejer}
    Determine whether the following series converges or diverges.
    $$
        \sum_{k=1}^{\infty} \sin\left( \frac{1}{k} \right)
    $$
\end{ejer}
\begin{proof}[Solution]
    Let
    $$
        a_k = \sin\left( \frac{1}{k} \right)
    $$
    and
    $$
        b_k = \frac{1}{k}
    $$
    for $ k \in \mathbb{N}$. Then,
    \begin{align*}
        \lim_{k \to \infty} \frac{a_k}{b_k}
            & = \lim_{k \to \infty} \frac{\sin\left( \frac{1}{k}\right)}{\frac{1}{k}} \\
            & = 1.
    \end{align*}
    Since 
    $$
    \lim_{k \to \infty} \frac{a_k}{b_k} = 1
    $$
    is positive, finite, and
    $$
        \sum_{k=1}^{\infty} \frac{1}{k}
    $$
    is divergent (harmonic series), we conclude that 
    $$
        \sum_{k=1}^{\infty} \sin\left( \frac{1}{k} \right)
    $$
    is divergent.
\end{proof}
\end{document}