% \documentclass[compacto,10pt]{aleph-notas}
\documentclass[compacto,10pt,comentarios]{aleph-notas}

% -- Paquetes adicionales
\usepackage{enumitem}
\usepackage{aleph-comandos}
\usepackage{parskip}
\usepackage{graphicx}
\usepackage{xfrac}
\usepackage{tikz}
\usepackage{etoolbox}
\usepackage[framemethod=tikz]{mdframed}
\DeclareFontFamily{U}{skulls}{}
\DeclareFontShape{U}{skulls}{m}{n}{ <-> skull }{}
\newcommand{\skull}{\text{\usefont{U}{skulls}{m}{n}\symbol{'101}}}
\def \ds{\displaystyle}
\def \dfx{\dfrac{d}{dx}}
\DeclareMathOperator{\arccot}{arccot}
\DeclareMathOperator{\arcsec}{arcsec}
\DeclareMathOperator{\arccsc}{arccsc}
\newcommand*\Heq{\ensuremath{\overset{\kern2pt LH}{=}}}

% -- Datos del libro
\institucion{Southwestern College}
\asignatura{MATH 251: Calculus II}
\tema{Infinite Series}
\autor{Jesús Pérez Cuarenta}
% \fecha{Fall 2024}

%% --> Logos de las guias
\logouno[4.5cm]{../Images/swc_logo}
\definecolor{colordef}{cmyk}{0.81, 0.62, 0.00, 0.22}

%% -- Solucion para alumnos
% \AtBeginEnvironment{proof}{\color{white}}

\begin{document}

\encabezado

\section*{Infinite Series}
\begin{mdframed}
    \center Learning Objectives \\
    \begin{itemize}
        \item Apply the Divergence Test.
        \item Apply the Integral Test.
        \item Apply the $p$-series Test.
    \end{itemize}
\end{mdframed}

One of the simplest tests determines whether an infinite series diverges.
\begin{teo}[Divergence Test]
    If
    $$
        \sum_{k=1}^{\infty} a_k
    $$
    converges, then 
    $$
        \lim_{k \to \infty} a_k = 0.
    $$
    Equivalently, if
    $$
        \lim_{k \to \infty} a_k \neq 0
    $$
    then the series diverges.
\end{teo}

\begin{ejer}
    Determine if the following series is divergent.
    $$
        \sum_{k=1}^{\infty} \left( \frac{2k^2 + 1}{3k^2 - 1} \right)
    $$
\end{ejer}
\begin{proof}[Solution]
    We have
    $$
        \lim_{k \to \infty} a_k = \lim_{k \to \infty} \frac{2k^2+1}{3k^2-1} = \frac{2}{3} \neq 0.
    $$
    Therefore, the infinite series
    $$
        \sum_{k=1}^{\infty} \left( \frac{2k^2 + 1}{3k^2 - 1} \right)
    $$
    is divergent.
\end{proof}

\begin{ejer}
    Determine if the following series is divergent.
    $$
        \sum_{k=1}^{\infty} \mathrm{e}^{k^2}
    $$
\end{ejer}
\begin{proof}[Solution]
    We have
    $$
        \lim_{k \to \infty} a_k = \lim_{k \to \infty} \mathrm{e}^{k^2} = \infty \neq 0.
    $$
    Therefore, the infinite series
    $$
        \sum_{k=1}^{\infty} \mathrm{e}^{k^2}
    $$
    is divergent.
\end{proof}

\begin{ejer}
    Determine if the following series is divergent.
    $$
        \sum_{k=1}^{\infty} \left( \frac{1}{5k^2 + 3} \right)
    $$
\end{ejer}
\begin{proof}[Solution]
    We have
    $$
        \lim_{k \to \infty} a_k = \lim_{k \to \infty} \left( \frac{1}{5k^2 + 3} \right) = 0.
    $$
    Therefore, the test is inconclusive.
\end{proof}

\begin{teo}[Integral Test]
    Suppose $f$ is a continuous, positive, decreasing function, for $x \geq 1$, and let
    $$
        a_k = f(k),
    $$
    for $k \in \mathbb{N}$. Then
    $$
        \sum_{k=1}^{\infty} a_k \text{ and } \int_{1}^{\infty} f(x) dx
    $$
    either both converge or both diverge. In the case of convergence, the value of the integral is not equal to the value of series.
\end{teo}

\begin{ejer}
    Use the integral test to determine whether the given series converges or diverges.
    $$
        \sum_{k=1}^{\infty} \frac{1}{k}
    $$
\end{ejer}
\begin{proof}[Solution]
    Let $f(x) = \frac{1}{x}$ be defined for $x \in \mathbb{R}$. We check three items.
    \begin{enumerate}
        \item Continuity \\
        We have that $f(x)$ is continuous over $[1, \infty)$.
        \item Positivity \\
        If $x \geq 1$ then $f(x) = \frac{1}{x} > 0$.
        \item Decreasing function \\
        We see that $f'(x) = -\frac{1}{x^2}$ and if $x \geq 1$ then $f'(x) < 0$ which implies $f$ is decreasing over $[1, \infty)$.
    \end{enumerate}
    Now we evaluate
    \begin{align*}
        \int_{1}^{\infty} f(x) ~ dx 
            & = \lim_{b \to \infty} \int_{1}^{b} \frac{1}{x} ~ dx  \\
            & = \lim_{b \to \infty} \left. \ln(x) \right\rvert_{1}^{b} \\
            & = \lim_{b \to \infty} \ln(b) - \ln(1) \\
            & = \infty.
    \end{align*}
    Since
    $$
        \int_{1}^{\infty} \frac{1}{x} ~ dx
    $$
    is divergent, we conclude
    $$
        \sum_{k=1}^{\infty} \frac{1}{k}
    $$
    is also divergent.
\end{proof}

\begin{ejer}
    Use the integral test to determine whether the given series converges or diverges.
    $$
        \sum_{k=1}^{\infty} \frac{1}{k^2}
    $$
\end{ejer}
\begin{proof}[Solution]
    Let $f(x) = \frac{1}{x^2}$ be defined for $x \in \mathbb{R}$. We check three items.
    \begin{enumerate}
        \item Continuity \\
        We have that $f(x)$ is continuous over $[1, \infty)$.
        \item Positivity \\
        If $x \geq 1$ then $f(x) = \frac{1}{x^2} > 0$.
        \item Decreasing function \\
        We see that $f'(x) = -\frac{2}{x^3}$ and if $x \geq 1$ then $f'(x) < 0$ which implies $f$ is decreasing over $[1, \infty)$.
    \end{enumerate}
    Now we evaluate
    \begin{align*}
        \int_{1}^{\infty} f(x) ~ dx 
            & = \lim_{b \to \infty} \int_{1}^{b} \frac{1}{x^2} ~ dx  \\
            & = \lim_{b \to \infty} \left. -\frac{1}{x} \right\rvert_{1}^{b} \\
            & = \lim_{b \to \infty} -\frac{1}{b} + \frac{1}{1} \\
            & = 1.
    \end{align*}
    Since
    $$
        \int_{1}^{\infty} \frac{1}{x^2} ~ dx
    $$
    is convergent, we conclude
    $$
        \sum_{k=1}^{\infty} \frac{1}{k^2}
    $$
    is also convergent.
\end{proof}

\begin{ejer}
    Use the integral test to determine whether the given series converges or diverges.
    $$
        \sum_{k=2}^{\infty} \frac{\ln(k)}{k}
    $$
\end{ejer}
\begin{proof}[Solution]
    Let $f(x) = \frac{\ln(x)}{x}$ be defined for $x \in \mathbb{R}$. We check three items.
    \begin{enumerate}
        \item Continuity \\
        We have that $f(x)$ is continuous over $[2, \infty)$.
        \item Positivity \\
        If $x \geq 2$ then $f(x) = \frac{\ln(x)}{x} > 0$ (since $f$ is the ratio of two positive values).
        \item Decreasing function \\
        We see that
        $$
            f'(x) = \frac{1-\ln(x)}{x^2}
        $$
        which has critical points for $x_c \in \{0, \mathrm{e}\}$. Now,
        \begin{align*}
            f'(\mathrm{e}^2)
                & = \frac{(1 - \ln(\mathrm{e}^{2}))}{(\mathrm{e}^{2})^2} \\
                & = -\frac{1}{\mathrm{e}^{4}} \\
                & < 0
        \end{align*}
        shows that $f$ is decreasing over $[\mathrm{e}, \infty)$.
    \end{enumerate}
    Thus,
    \begin{align*}
        \int_{\mathrm{e}}^{\infty} f(x) ~ dx 
            & = \lim_{b \to \infty} \int_{\mathrm{e}}^{b} \frac{\ln(x)}{x} ~ dx  \\
            & = \lim_{b \to \infty} \left. \frac{1}{2} \left( \ln(x) \right)^{2} \right\rvert_{\mathrm{e}}^{b} \\
            & = \lim_{b \to \infty} \frac{1}{2} \left( \ln(b)^{2} - \ln(\mathrm{e})^2 \right) \\
            & = \infty.
    \end{align*}
    Since
    $$
        \int_{\mathrm{e}}^{\infty} \frac{\ln(x)}{x} ~ dx
    $$
    is divergent, we conclude
    $$
        \sum_{k=2}^{\infty} \frac{\ln(k)}{k}
    $$
    is also divergent.
\end{proof}

\begin{teo}[Convergence of the $p$-series]
    The \textbf{$p$-series}
    $$
        \sum_{k=1}^{\infty} \frac{1}{k^{p}}
    $$
    converges for $p > 1$ and diverges for $p \leq 1$.
\end{teo}

\begin{ejer}
    Determine whether the series converges or diverges.
    $$
        \sum_{k=1}^{\infty} \frac{1}{k^2}
    $$
\end{ejer}
\begin{proof}[Solution]
    The infinite series is convergent since it is a $p$-series with
    $$
        p = 2 > 1.
    $$
\end{proof}

\begin{ejer}
    Determine whether the series converges or diverges.
    $$
        \sum_{k=1}^{\infty} \frac{1}{\sqrt{k}}
    $$
\end{ejer}
\begin{proof}[Solution]
    The infinite series is divergent since it is a $p$-series with
    $$
        p = \frac{1}{2} \leq 1.
    $$
\end{proof}

\begin{ejer}
    Determine whether the series converges or diverges.
    $$
        \sum_{k=1}^{\infty} k^{-1.001}
    $$
\end{ejer}
\begin{proof}[Solution]
    The infinite series is convergent since it is a $p$-series with
    $$
        p = 1.001 > 1.
    $$
\end{proof}

\begin{ejer}
    Determine whether the series converges or diverges.
    $$
        \sum_{k=1}^{\infty} \frac{\sqrt[3]{k^2}}{5\sqrt[4]{k^3}}
    $$
\end{ejer}
\begin{proof}[Solution]
    We rewrite the infinite series
    \begin{align*}
        \sum_{k=1}^{\infty} \frac{\sqrt[3]{k^2}}{5\sqrt[4]{k^3}}
            & = \sum_{k=1}^{\infty} \frac{1}{5} \frac{k^{\frac{2}{3}}}{k^{\frac{3}{4}}} \\
            & = \sum_{k=1}^{\infty} \frac{1}{5} \frac{1}{k^\frac{1}{12}} \\
    \end{align*}
    The infinite series is divergent since it is a $p$-series with 
    $$
        p = \frac{1}{12} \leq 1.
    $$
\end{proof}
\end{document}