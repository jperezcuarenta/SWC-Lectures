% \documentclass[compacto,10pt]{aleph-notas}
\documentclass[compacto,10pt,comentarios]{aleph-notas}

% -- Paquetes adicionales
\usepackage{enumitem}
\usepackage{aleph-comandos}
\usepackage{parskip}
\usepackage{graphicx}
\usepackage{xfrac}
\usepackage{tikz}
\usepackage{etoolbox}
\usepackage[framemethod=tikz]{mdframed}
\DeclareFontFamily{U}{skulls}{}
\DeclareFontShape{U}{skulls}{m}{n}{ <-> skull }{}
\newcommand{\skull}{\text{\usefont{U}{skulls}{m}{n}\symbol{'101}}}
\def \ds{\displaystyle}
\def \dfx{\dfrac{d}{dx}}
\DeclareMathOperator{\arccot}{arccot}
\DeclareMathOperator{\arcsec}{arcsec}
\DeclareMathOperator{\arccsc}{arccsc}
\newcommand*\Heq{\ensuremath{\overset{\kern2pt LH}{=}}}

% -- Datos del libro
\institucion{Southwestern College}
\asignatura{MATH 251: Calculus II}
\tema{Infinite Sequences}
\autor{Jesús Pérez Cuarenta}
% \fecha{Fall 2024}

%% --> Logos de las guias
\logouno[4.5cm]{../Images/swc_logo}
\definecolor{colordef}{cmyk}{0.81, 0.62, 0.00, 0.22}

%% -- Solucion para alumnos
\AtBeginEnvironment{proof}{\color{white}}

%% 
% https://www.emathhelp.net/en/calculators/algebra-2/partial-fraction-decomposition-calculator/

\begin{document}

\encabezado

\section*{Infinite Sequences}
\begin{mdframed}
    \center Learning Objectives \\
    \begin{itemize}
        \item Define a sequence explicitly and recursively.
        \item Determine whether a sequence is convergent or divergent.
        \item Define monotonic and bounded sequences.
    \end{itemize}
\end{mdframed}

\begin{defi}[Sequences]
    A \textbf{sequence} $\{ a_{n} \}$ is an ordered list of numbers of the form
    $$
        \{ a_1, a_2, a_3, \ldots, a_n, \ldots \}.
    $$
    A sequence may be generated by a \textbf{recurrence relation} 
    of the form 
    $$
        a_{n+1} = f(a_n)
    $$
    for $n \in \mathbb{N}$, where $a_1$ is given.
    A sequence may also be defined with an \textbf{explicit formula} of the form 
    $$
        a_n = f(n)
    $$
    for $n \in \mathbb{N}$.
\end{defi}

\begin{ejer}
    Let $a_n = 3n-2$. Write the first five terms of the sequence $\{a_n\}_{n=1}^{\infty}$.
\end{ejer}
\begin{proof}[Solution]
    We have
    \begin{align*}
        a_1 & = 3(1) - 2 = 1 \\
        a_2 & = 3(2) - 2 = 4 \\ 
        a_3 & = 3(3) - 2 = 7 \\
        a_4 & = 3(4) - 2 = 10 \\
        a_5 & = 3(5) - 2 = 13 ~ .
    \end{align*}    
\end{proof}

\begin{ejer}
    Define
    $$
        a_n = (-1)^{n+1} \frac{n}{n+1}.
    $$
    Write the first five terms of the sequence $\{a_n\}_{n=1}^{\infty}$.
\end{ejer}
\begin{proof}[Solution]
    We have
    \begin{align*}
        a_1 & = (-1)^{1+1} \frac{1}{1+1} = \frac{1}{2} \\
        a_2 & = (-1)^{2+1} \frac{2}{2+1} = -\frac{2}{3} \\ 
        a_3 & = (-1)^{3+1} \frac{3}{3+1} = \frac{3}{4} \\
        a_4 & = (-1)^{4+1} \frac{4}{4+1} = -\frac{4}{5} \\ 
        a_5 & = (-1)^{5+1} \frac{5}{5+1} = \frac{5}{6} ~ .
    \end{align*}    
\end{proof}

\begin{ejer}
    Define 
    $$
        a_{n+1} = 2a_{n} - 3
    $$
    where $a_1 = 4$. Write the first five terms of the sequence $a_n$.
\end{ejer}
\begin{proof}
    Since $a_1$ is given, we can evaluate $a_2$. We have,
    $$
        a_2 = 2a_{1} - 3 = 2 \cdot 4 - 3 = 5.
    $$
    Knowing $a_2$, we can evaluate $a_3$. Namely,
    $$
        a_3 = 2a_2 - 3 = 2 \cdot 5 - 3 = 7 ~ .
    $$
    Similarly, $a_4$ and $a_5$ produce
    \begin{align*}
        a_4 & = 2a_3 - 3 = 2 \cdot 7 - 3 = 11 \\
        a_5 & = 2a_4 - 3 = 2 \cdot 11 - 3 = 19 ~ .
    \end{align*}
    Thus, the first five terms of the sequence are expressed by
    $$
        \{a_n\}_{n=1}^{5} = \{4, 5, 7, 11, 19\}.
    $$
\end{proof}

Some sequences can be defined with a special product which relies on the \textbf{factorial} operator. We now define such operation and work out a few examples.
\begin{defi}[Factorial]
    The \textbf{factorial} of a non-negative integer $n$, denoted by $n!$, is the product of all positive integers less than or equal to $n$.
    We define
    $$
        n! := n \cdot (n-1) \cdot (n-2) \cdot \ldots \cdot 3 \cdot 2 \cdot 1
    $$
    and take the convention that $0! = 1$.
\end{defi}

\begin{ejer}
    Simplify
    $$
        \frac{9!}{7!} ~ .
    $$
\end{ejer}
\begin{proof}[Solution]
    We have
    $$
        \frac{9!}{7!} = \frac{9 \cdot 8 \cdot 7!}{7!}  = \frac{9 \cdot 8}{1} = 72 ~ .
    $$
\end{proof}

\begin{ejer}
    Simplify
    $$
        \frac{12!}{3! \cdot 10!} ~ .
    $$
\end{ejer}
\begin{proof}[Solution]
    We have
    $$
        \frac{12!}{3! \cdot 10!} = \frac{12 \cdot 11 \cdot 10!}{3! \cdot 10!} = \frac{12 \cdot 11}{3 \cdot 2} = 22 ~ .
    $$
\end{proof}

\begin{ejer}
    Simplify
    $$
        \frac{(n + 1)!}{n!} ~ .
    $$
\end{ejer}
\begin{proof}[Solution]
    We have
    $$
        \frac{(n + 1)!}{n!} = \frac{(n+1) \cdot n!}{n!} = n+1~.
    $$
\end{proof}

\begin{ejer}
    Simplify
    $$
        \frac{(3n-1)!}{(3n+1)!} ~ .
    $$
\end{ejer}
\begin{proof}[Solution]
    We have
    $$
        \frac{(3n-1)!}{(3n+1)!} = \frac{(3n-1)!}{(3n+1) \cdot (3n) \cdot (3n-1)!} = \frac{1}{(3n + 1)(3n)} ~ .
    $$
\end{proof}

\begin{ejer}
    Write the $n$th term formula for the sequence
    $$
        \{a_n\}_{n=1}^{\infty} = \left\{ \frac{2}{1}, -\frac{4}{3}, \frac{8}{5}, -\frac{16}{7}, \frac{32}{9}, \ldots \right\}
    $$
\end{ejer}
\begin{proof}[Solution]
    We make the observation that the numerator can always be expressed as a power of $2$, the denominator is always an odd number, and the elements of the sequence alternate between positive and negative values. Thus,
    $$
        \{a_n\}_{n=1}^{\infty} = \left\{ \left(-1\right)^{n+1} \frac{2^{n}}{2n-1} \right\}_{n=1}^{\infty} ~ .
    $$
\end{proof}

\begin{ejer}
    Write the $n$th term formula for the sequence
    $$
        \{a_n\}_{n=1}^{\infty} = \left\{ 3, 7, 11, 15, \ldots \right\}
    $$
\end{ejer}
\begin{proof}[Solution]
    Note that any two consecutive terms differ by 4. Hence,
    $$
        \{a_n\}_{n=1}^{\infty} = \left\{ 4n-1 \right\}_{n=1}^{\infty} ~ .
    $$
\end{proof}

\begin{ejer}
    Write the $n$th term formula for the sequence
    $$
        \{a_n\}_{n=1}^{\infty} = \left\{ 2, -1, \frac{1}{2}, -\frac{1}{4}, \ldots \right\}
    $$
\end{ejer}
\begin{proof}[Solution]
    Note that the $n$th term is the previous term divided by $2$, with alternating signs. Hence,
    $$
        \{a_n\}_{n=1}^{\infty} = \left\{ (-1)^{n+1} \frac{1}{2^{n-2}} \right\}_{n=1}^{\infty} ~ .
    $$
\end{proof}

\begin{ejer}
    Write the $n$th term formula for the sequence
    $$
        \{a_n\}_{n=0}^{\infty} = \left\{ 1, x, \frac{x^{2}}{2}, \frac{x^{3}}{6}, \frac{x^{4}}{24}, \frac{x^{5}}{120}, \ldots \right\}
    $$
\end{ejer}
\begin{proof}[Solution]
    Note that denominator of the $n$th term follows the pattern of the factorial operator. On the other hand, the powers of $x$ increase by 1. Hence,
    $$
        \{a_n\}_{n=0}^{\infty} = \left\{ \frac{x^{n}}{n!} \right\}_{n=0}^{\infty} ~ .
    $$
\end{proof}

\begin{defi}[Limit of a Sequence]
    If the terms of a sequence $\{a_n\}$ approach a unique number $L$ as $n$ increases then we say that
    $$
        \lim_{n \to \infty} a_n = L
    $$
    exists, and the sequence \textbf{converges} to $L$. If the terms of the sequence do not approach a single number as $n$ increases, the sequence has no limit, and the sequence \textbf{diverges}.
\end{defi}

\begin{ejer}
    Find the limit, if it exists, for the sequence
    $$
        \{a_n\}_{n=1}^{\infty} = \left\{\frac{n}{n+1}\right\}_{n=1}^{\infty} ~ .
    $$
\end{ejer}
\begin{proof}[Solution]
    Note that
    $$
        \lim_{n \to \infty} \frac{n}{n+1} = \lim_{n \to \infty} \frac{1}{1 + \frac{1}{n}} = 1 ~ .
    $$
    Hence, the sequence $\{a_n\}$ is convergent.
\end{proof}

\begin{ejer}
    Find the limit, if it exists, for the sequence
    $$
        \{a_n\}_{n=1}^{\infty} = \left\{ (-2)^{n} \right\}_{n=1}^{\infty} ~ .
    $$
\end{ejer}
\begin{proof}[Solution]
    Note that
    $$
        \left\{ (-2)^{n} \right\}_{n=1}^{\infty} = \left\{ -2, 4, -8, 16, -32, \ldots \right\} ~ .
    $$
    Hence, the sequence $\{a_n\}$ is divergent.
\end{proof}

\begin{ejer}
    Find the limit, if it exists, for the sequence
    $$
        \{a_n\}_{n=1}^{\infty} = \left\{ (-2)^{n} \right\}_{n=1}^{\infty} ~ .
    $$
\end{ejer}
\begin{proof}[Solution]
    Note that
    $$
        \left\{ (-2)^{n} \right\}_{n=1}^{\infty} = \left\{ -2, 4, -8, 16, -32, \ldots \right\} ~ .
    $$
    Taking
    $$
        \lim_{n \to \infty} a_n = \lim_{n \to \infty} (-1)^{n} 2^{n}
    $$ 
    we see that the limit does not exist. Hence, the sequence $\{a_n\}$ is divergent.
\end{proof}

\begin{ejer}
    Find the limit, if it exists, for the sequence
    $$
        \{a_n\}_{n=1}^{\infty} = \left\{ \left(\frac{1}{3}\right)^{n} \right\}_{n=1}^{\infty} ~ .
    $$
\end{ejer}
\begin{proof}[Solution]
    Note that
    $$
        \left\{ \left(\frac{1}{3}\right)^{n} \right\}_{n=1}^{\infty} = \left\{ \frac{1}{3}, \frac{1}{9}, \frac{1}{27}, \ldots \right\} ~ .
    $$
    Taking the limit as $n$ goes to infinity shows that $a_n$ tends to zero, i.e.,
    $$
        \lim_{n \to \infty} a_n = \lim_{n \to \infty} \left(\frac{1}{3}\right)^{n} = 0 ~ .
    $$ 
    Hence, the sequence $\{a_n\}$ is convergent.
\end{proof}

\begin{ejer}
    Find the limit, if it exists, for the sequence
    $$
        \{a_n\}_{n=1}^{\infty} = \left\{ 2 + (-1)^{n} \right\}_{n=1}^{\infty} ~ .
    $$
\end{ejer}
\begin{proof}[Solution]
    The sequence $\{a_n\}$ alternates between two discrete values for every $n \geq 1$. Hence, the sequence is divergent.
\end{proof}

\begin{teo}[Limits of Sequences from Limits of Functions]
    Suppose $f$ is a function such that $f(n) = a_n$ for positive integers $n$. If
    $$
        \lim_{x \to \infty} f(x) = L,
    $$
    then the limit of the sequence $\{a_n\}$ is also $L$, where $L$ may be $\pm \infty$.
\end{teo}
\begin{ejer}
    Find the limit, if it exists, for the sequence
    $$
        \{a_n\}_{n=1}^{\infty} = \left\{ \left( 1 + \frac{1}{n} \right)^{n}  \right\}_{n=1}^{\infty} ~ .
    $$
\end{ejer}
\begin{proof}[Solution]
    We have
    \begin{align*}
        \lim_{n \to \infty} a_n & = \lim_{n \to \infty} \left( 1 + \frac{1}{n} \right)^{n} \\
        & = \lim_{n \to \infty} \exp{ \left( \ln\left(\left( 1 + \frac{1}{n} \right)^{n}\right) \right)} \\
        & = \lim_{n \to \infty} \exp{ \left( n \ln\left(1 + \frac{1}{n} \right) \right)} \\
        & = \lim_{n \to \infty} \exp{ \left( \frac{\ln\left(1 + \frac{1}{n} \right)}{\frac{1}{n}} \right)} ~ .
        % & \Heq \lim_{n \to \infty} \exp{ \left( \frac{\ln\left(1 + \frac{1}{n} \right)}{\frac{1}{n}} \right)} \\
    \end{align*}
    Now we extend the domain of $a_n$ by letting
    $$
        f(x) = \frac{\ln\left( 1 + \frac{1}{x} \right)}{\frac{1}{x}}
    $$
    be defined for $x \in \mathbb{R}$. Since derivatives are admissible now, we get
    \begin{align*}
        \lim_{n \to \infty} \exp{ \left( \frac{\ln\left(1 + \frac{1}{n} \right)}{\frac{1}{n}} \right)}
        & = \lim_{x \to \infty} \exp{ \left( \frac{\ln\left(1 + \frac{1}{x} \right)}{\frac{1}{x}} \right)} \\
        & \Heq \lim_{x \to \infty} \exp{ \left( \frac{\frac{d}{dx} \ln\left(1 + \frac{1}{x} \right)}{ \frac{d}{dx} \frac{1}{x}} \right)} \\
        & = \lim_{x \to \infty} \exp{ \left( \frac{ \left(1 + \frac{1}{x} \right)^{-1} \frac{d}{dx} \left(1 + \frac{1}{x} \right)}{ \frac{d}{dx} \frac{1}{x}} \right)} \\
        & = \lim_{x \to \infty} \exp{ \left( \frac{ 1}{\left(1 + \frac{1}{x} \right)} \right)} \\ 
        & = \exp{\left(1\right)}.
    \end{align*}
    Therefore, we get that the sequence $a_n$ converges to Euler's constant. In other words,
    $$
        \lim_{n \to \infty} a_n = \lim_{n \to \infty} \left( 1 + \frac{1}{n} \right)^{n} = \mathrm{e} .
    $$
\end{proof}

Because of the correspondence between limits of sequences and limits of functions at infinity, we have the following properties that are analogous to those for functions.
\begin{teo}[Limit Laws for Sequences]
    Assume the sequences $\{a_n\}$ and $\{b_n\}$ have limits $A$ and $B$ as $n$ tends to infinity, respectively. Then,
    \begin{itemize}
        \item
        $$
            \lim_{n \to \infty} \left(a_n \pm b_n \right) = A \pm B
        $$
        \item
        $$
            \lim_{n \to \infty} c \cdot a_n = c \cdot A \quad (c \in \mathbb{R})
        $$
        \item
        $$
            \lim_{n \to \infty} a_n \cdot b_n = A \cdot B
        $$
        \item
        $$
            \lim_{n \to \infty} \frac{a_n}{b_n} = \frac{A}{B} \quad (B \neq 0)
        $$
    \end{itemize}
\end{teo}

\begin{defi}[Terminology for Sequences]
    \begin{itemize}
        \item The sequence $\{a_n\}$ is \textbf{increasing} if $a_{n+1} > a_n$ for $n \in \mathbb{N}$. For example
        $$
            \{a_n\} = \{0, 1, 2, 3, \ldots \}.
        $$
        \item The sequence $\{a_n\}$ is \textbf{nondecreasing} if $a_{n+1} \geq a_n$ for $n \in \mathbb{N}$. For example
        $$
            \{a_n\} = \{1, 1, 2, 2, 3, 3, \ldots \}.
        $$
        \item The sequence  $\{a_n\}$ is \textbf{decreasing} if $a_{n+1} < a_n$ for $n \in \mathbb{N}$. For example
        $$
            \{a_n\} = \{2, 1, 0, -1 \ldots \}.
        $$
        \item  The sequence $\{a_n\}$ is \textbf{nonincreasing} if $a_{n+1} \leq a_n$ for $n \in \mathbb{N}$. For example
        $$
            \{a_n\} = \{0, -1, -1, -2, -2,\ldots \}.
        $$
        \item The sequence $\{a_n\}$ is \textbf{monotonic} if it is either nonincreasing or nondecreasing.
        \item The sequence $\{a_n\}$ is \textbf{bounded above} if there is a number $M$ such that
        $$
            a_n \leq M
        $$
        for all $n > N$ where $N \in \mathbb{N}$.
        \item The sequence $\{a_n\}$ is \textbf{bounded below} if there is a number $N$ such that
        $$
            a_n \geq N
        $$
        for all $n > N$ where $N \in \mathbb{N}$.
        \item The sequence $\{a_n\}$ is \textbf{bounded} if it is bounded both from above and below.
    \end{itemize}
\end{defi}

\begin{ejer}
    Show that 
    $$
        \{a_n\}_{n=1}^{\infty} = \left\{ \frac{1}{n^{2} + 1} \right\}_{n=1}^{\infty}
    $$
    is a decreasing sequence.
\end{ejer}
\begin{proof}[Solution]
    Let $n \in \mathbb{N}$. We have,
    \begin{align*}
        1 & > 0 \\
        \implies n + 1 & > n \\
        \implies (n+1)^2 & > n^2 \\
        \implies (n+1)^2 + 1 & > n^2 + 1 \\
        \implies \frac{1}{(n+1)^{2} + 1} & < \frac{1}{n^2 + 1} \\
        \implies a_{n+1} & < a_{n} ~.
    \end{align*}
    By definition, we have that $\{a_n\}$ is decreasing.
\end{proof}

\begin{ejer}
    Determine whether the sequence
    $$
        \{a_{n}\}_{n=1}^{\infty} = \left\{ 2 + \mathrm{e}^{-\frac{n}{3}} \right\}_{n=1}^{\infty}
    $$
    is bounded.
\end{ejer}
\begin{proof}[Solution]
    Let $n \in \mathbb{N}$. We have
    \begin{align*}
        \mathrm{e}^{-\frac{n}{3}} & > 0 \\
        \implies 2 + \mathrm{e}^{-\frac{n}{3}} & > 2
    \end{align*}
    which implies $\{a_n\}$ is bounded from below by $2$. Now we show that $\{a_n\}$ is a decreasing sequence. Notice that
    \begin{align*}
        0 & <  1 \\
        \implies n & < n + 1 \\
        \implies -n & > - (n + 1) \\
        \implies -\frac{n}{3} & > - \frac{n + 1}{3} \\
        \implies \mathrm{e}^{-\frac{n}{3}} & > \mathrm{e}^{- \frac{n + 1}{3}} \\
        \implies 2 + \mathrm{e}^{-\frac{n}{3}} & >  2 + \mathrm{e}^{- \frac{n + 1}{3}} \\
        \implies a_n & >  a_{n+1}.
    \end{align*}
    We conclude the maximum value in the sequence $\{a_n\}$ is $a_1$. Hence, $\{a_n\}$ is bounded from above. It follows that $\{a_n\}$ is bounded.
\end{proof}

\begin{ejer}
    Determine whether the sequence
    $$
        \{a_n\}_{n=1}^{\infty} = \left\{ \arctan(\pi n)\right\}_{n=1}^{\infty}
    $$
    is bounded.
\end{ejer}
\begin{proof}[Solution]
    Let $f(x) = \arctan(\pi x)$ be defined over the real numbers. Differentiating with respect to $x$ yields
    $$
        f'(x) = \frac{\pi}{1+(\pi x)^2} > 0.
    $$
    Since $f'(x)$ is positive everywhere we conclude that $\{a_n\}$ is a strictly increasing sequence. This implies $a_1$ is the minimum value attained by elements of $\{a_n\}$. Hence, $\{a_n\}$ is bounded from below. Moreover,
    $$
        \lim_{n \to \infty} a_n = \lim_{n \to \infty} \arctan(\pi n) = \frac{\pi}{2}
    $$
    shows that $\{ a_n\}$ is bounded from above. Therefore, $\{a_n\}$ is bounded.
\end{proof}

\begin{ejer}
    Determine whether the sequence
    $$
        \{a_n\}_{n=1}^{\infty} = \left\{ \sin\left( \frac{\pi}{2} n \right) \right\}_{n=1}^{\infty}
    $$
    is bounded.
\end{ejer}
\begin{proof}[Solution]
    We have that $\{ a_n \} = \{1, 0, -1, 0, 1, 0, -1, 0, \ldots \}$ is bounded since for any $n \in \mathbb{N}$
    $$
        -1 \leq \sin\left(\frac{\pi}{2}n \right) \leq 1.
    $$ 
\end{proof}
\end{document}