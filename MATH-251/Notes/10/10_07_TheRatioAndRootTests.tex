% \documentclass[compacto,10pt]{aleph-notas}
\documentclass[compacto,10pt,comentarios]{aleph-notas}

% -- Paquetes adicionales
\usepackage{enumitem}
\usepackage{aleph-comandos}
\usepackage{parskip}
\usepackage{graphicx}
\usepackage{xfrac}
\usepackage{tikz}
\usepackage{etoolbox}
\usepackage[framemethod=tikz]{mdframed}
\DeclareFontFamily{U}{skulls}{}
\DeclareFontShape{U}{skulls}{m}{n}{ <-> skull }{}
\newcommand{\skull}{\text{\usefont{U}{skulls}{m}{n}\symbol{'101}}}
\def \ds{\displaystyle}
\def \dfx{\dfrac{d}{dx}}
\DeclareMathOperator{\arccot}{arccot}
\DeclareMathOperator{\arcsec}{arcsec}
\DeclareMathOperator{\arccsc}{arccsc}
\newcommand*\Heq{\ensuremath{\overset{\kern2pt LH}{=}}}

% -- Datos del libro
\institucion{Southwestern College}
\asignatura{MATH 251: Calculus II}
\tema{Ratio and Root Tests}
\autor{Jesús Pérez Cuarenta}
% \fecha{Fall 2024}

%% --> Logos de las guias
\logouno[4.5cm]{../Images/swc_logo}
\definecolor{colordef}{cmyk}{0.81, 0.62, 0.00, 0.22}

%% -- Solucion para alumnos
\AtBeginEnvironment{proof}{\color{white}}

\begin{document}

\encabezado
\section*{Ratio and Root Tests}
\begin{mdframed}
    \center Learning Objectives \\
    \begin{itemize}
        \item Define the ratio and root tests.
        \item Use the ratio test to determine whether a series converges.
        \item Use the root test to determine whether a series converges.
    \end{itemize}
\end{mdframed}

\begin{defi}[Ratio Test]
    Let
    $$
        S = \sum_{k=1}^{\infty} a_k
    $$
    be an infinite series.
    \begin{itemize}
        \item If
            $$
                \lim_{k \to \infty} \left| \frac{a_{k + 1}}{a_k} \right| < 1
            $$
            then the series $S$ is convergent.
        \item If
            $$
                \lim_{k \to \infty} \left| \frac{a_{k+1}}{a_k} \right| > 1
            $$
            or
            $$
                \lim_{k \to \infty} \left| \frac{a_{k+1}}{a_k} \right| = \infty
            $$
            then the series $S$ is divergent.
        \item If
            $$
                \lim_{k \to \infty} \left| \frac{a_{k+1}}{a_k}\right| = 1
            $$
            the ratio test is inconclusive. 
    \end{itemize}
\end{defi}

\begin{ejer}
    Determine whether the infinite series converges or diverges.
    $$
        \sum_{k=1}^{\infty} (-1)^{k} \frac{k^2}{3^k}
    $$
\end{ejer}
\begin{proof}[Solution]
    Let $k \in \mathbb{N}$ and define
    $$
        a_k = (-1)^{k} \frac{k^2}{3^{k}}.
    $$
    Now,
    \begin{align*}
        \lim_{k \to \infty} \left| \frac{a_{k+1}}{a_k} \right| 
        & = \lim_{k \to \infty} \frac{\frac{(k + 1)^2}{3^{(k + 1)}}}{\frac{k^2}{3^{k}}} \\
        & = \lim_{k \to \infty} \frac{3^{k} (k+1)^2}{3\cdot 3^{k} k^2} \\
        & = \lim_{k \to \infty} \frac{1}{3} \cdot \frac{k^2+2k+1}{k^2} \\
        & = \frac{1}{3}.
    \end{align*}
    Therefore,
    $$
        \sum_{k=1}^{\infty} (-1)^{k} \frac{k^2}{3^k}
    $$
    converges (absolutely) by the ratio test.
\end{proof}

\begin{ejer}
    Determine whether the infinite series converges or diverges.
    $$
        \sum_{k=1}^{\infty} \frac{k!}{5^k}
    $$
\end{ejer}
\begin{proof}[Solution]
    Let $k \in \mathbb{N}$ and define
    $$
        a_k = \frac{k!}{5^{k}}.
    $$
    Now,
    \begin{align*}
        \lim_{k \to \infty} \left| \frac{a_{k+1}}{a_k} \right| 
        & = \lim_{k \to \infty} \frac{\frac{(k+1)!}{5^{(k + 1)}}}{\frac{k!}{5^{k}}} \\
        & = \lim_{k \to \infty} \frac{5^{k}(k+1)!}{5 \cdot 5^{k}k!} \\
        & = \lim_{k \to \infty} \frac{1}{5} \cdot \frac{(k+1)k!}{k!} \\
        & = \lim_{k \to \infty} \frac{1}{5} \cdot (k+1) \\
        & = \infty.
    \end{align*}
    Therefore,
    $$
    \sum_{k=1}^{\infty} \frac{k!}{5^k}
    $$
    diverges by the ratio test.
\end{proof}

\begin{ejer}
    Determine whether the infinite series converges or diverges.
    $$
        \sum_{k=0}^{\infty} \frac{(-1)^k}{k^2+1}
    $$
\end{ejer}
\begin{proof}[Solution]
    Let $k \in \mathbb{N}$ and define
    $$
        a_k = (-1)^k \frac{1}{k^2+1}.
    $$
    Now,
    \begin{align*}
        \lim_{k \to \infty} \left| \frac{a_{k+1}}{a_k} \right| 
        & = \lim_{k \to \infty} \frac{\frac{1}{(k+1)^2+1}}{\frac{1}{k^2+1}} \\
        & = \lim_{k \to \infty} \frac{k^2+1}{(k+1)^2+1} \\
        & = \lim_{k \to \infty} \frac{k^2+1}{k^2 + 2k + 2} \\
        & = 1.
    \end{align*}
    Therefore, the ratio test is inconclusive.
\end{proof}

\begin{ejer}
    Determine whether the infinite series converges or diverges.
    $$
        \sum_{k=1}^{\infty} (-1)^{k+1} \frac{k^2+4}{\mathrm{e}^{k}}
    $$
\end{ejer}
\begin{proof}[Solution]
    Let $k \in \mathbb{N}$ and define
    $$
        a_k = (-1)^{k+1} \frac{k^2+4}{\mathrm{e}^{k}}.
    $$
    Now,
    \begin{align*}
        \lim_{k \to \infty} \left| \frac{a_{k+1}}{a_k} \right| 
        & = \lim_{k \to \infty} \frac{\frac{(k+1)^2+4}{\mathrm{e}^{(k+1)}}}{\frac{k^2+4}{\mathrm{e}^{k}}} \\
        & = \lim_{k \to \infty} \frac{\mathrm{e}^{k} (k^2 + 2k + 5)}{\mathrm{e} \cdot \mathrm{e}^{k} (k^2 + 4)} \\
        & = \lim_{k \to \infty} \frac{1}{\mathrm{e}} \frac{k^2 + 2k + 5}{k^2 + 4} \\
        & = \frac{1}{\mathrm{e}}.
    \end{align*}
    Therefore,
    $$
        \sum_{k=1}^{\infty} (-1)^{k+1} \frac{k^2+4}{\mathrm{e}^{k}}
    $$
    converges (absolutely) by the ratio test.
\end{proof}

\begin{defi}[Root Test]
    Let 
    $$
        S = \sum_{k=1}^{\infty} a_k.
    $$
    \begin{itemize}
        \item If 
        $$
            \lim_{n\to\infty}\sqrt[k]{|a_k|} < 1
        $$ then $S$ is convergent.
        \item If
        $$
            \lim_{k \to \infty} \sqrt[k]{|a_k|} > 1
        $$ 
        or 
        $$
            \lim_{k \to \infty} \sqrt[k]{|a_k|}=\infty
        $$
        then $S$ is divergent.
        \item
        If 
            $$
                \lim_{k \to \infty} \sqrt[k]{|a_k|}=1
            $$
        the root test is inconclusive. 
\end{itemize}
\end{defi}

\begin{ejer}
    Determine whether the infinite series converges or diverges.
    $$
        \sum_{k=1}^{\infty} \left( \frac{2k + 3}{3 k + 2} \right) ^ {k}
    $$
\end{ejer}
\begin{proof}[Solution]
    Let $k \in \mathbb{N}$ and
    $$
        a_k = \left( \frac{2k + 3}{3 k + 2} \right) ^ {k}.
    $$
    Then,
    \begin{align*}
        \lim_{k \to \infty} \sqrt[k]{|a_k|}
        & = \lim_{k \to \infty} \sqrt[k]{\left( \frac{2k + 3}{3 k + 2} \right) ^ {k}} \\
        & = \lim_{k \to \infty} \frac{2k + 3}{3 k + 2} \\
        & = \frac{2}{3}.
    \end{align*}
    Therefore, 
    $$
        \sum_{k=1}^{\infty} \left( \frac{2k + 3}{3 k + 2} \right) ^ {k}
    $$
    converges by the root test.
\end{proof}

\begin{ejer}
    Determine whether the infinite series converges or diverges.
    $$
        \sum_{k=1}^{\infty} \frac{k ^ {k}}{3 ^ {2k + 1}}
    $$
\end{ejer}
\begin{proof}[Solution]
    Let $k \in \mathbb{N}$ and
    $$
        a_k = \frac{k ^ {k}}{3 ^ {2k + 1}}.
    $$
    Then,
    \begin{align*}
        \lim_{k \to \infty} \sqrt[k]{|a_k|}
        & = \lim_{k \to \infty} \sqrt[k]{\frac{k ^ {k}}{3 ^ {2k + 1}}} \\
        & = \lim_{k \to \infty} \frac{k}{3^{2 + \frac{1}{k}}} \\
        & = \frac{\infty}{9} \\
        & = \infty.
    \end{align*}
    Therefore,
    $$
        \sum_{k=1}^{\infty} \frac{k ^ {k}}{3 ^ {2k + 1}}
    $$
    is divergent (by the root test).
\end{proof}

\end{document}