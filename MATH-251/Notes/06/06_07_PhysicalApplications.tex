% \documentclass[compacto,10pt]{aleph-notas}
\documentclass[compacto,10pt,comentarios]{aleph-notas}

% -- Paquetes adicionales
\usepackage{enumitem}
\usepackage{aleph-comandos}
\usepackage{parskip}
\usepackage{graphicx}
\usepackage{xfrac}
\usepackage{tikz}
\usepackage{etoolbox}
% \AtBeginEnvironment{proof}{\color{white}}
\usepackage[framemethod=tikz]{mdframed}
\DeclareFontFamily{U}{skulls}{}
\DeclareFontShape{U}{skulls}{m}{n}{ <-> skull }{}
\newcommand{\skull}{\text{\usefont{U}{skulls}{m}{n}\symbol{'101}}}
\def \ds{\displaystyle}
\def \dfx{\dfrac{d}{dx}}
\DeclareMathOperator{\arccot}{arccot}
\DeclareMathOperator{\arcsec}{arcsec}
\DeclareMathOperator{\arccsc}{arccsc}

% -- Datos del libro
\institucion{Southwestern College}
\asignatura{MATH 251: Calculus II}
\tema{Physical Applications}
\autor{Jesús Pérez Cuarenta}
% \fecha{Fall 2024}

%% --> Logos de las guias
\logouno[4.5cm]{../Images/swc_logo}
\definecolor{colordef}{cmyk}{0.81, 0.62, 0.00, 0.22}

\begin{document}

\encabezado

\section*{Physical Applications}
We conclude the sixth chapter with problems from physics and engineering. The physical themes in these problems are mass, work, and force.
\begin{defi}[\textbf{Mass of a One-Dimensional Object}]
    Suppose a thin bar or wire is represented by the interval $a \leq x \leq b$ with a density function $\rho$ (with units of mass per length). The \textbf{mass} of the object is
    $$
        m = \int_{a}^{b} \rho(x) ~ dx.
    $$
\end{defi}

\begin{defi}[\textbf{Work}]
    The work done by a variable force $F$ moving an object along a line from $x = a$ to $x = b$ in the direction of the force is
    $$
        W = \int_{a}^{b} F(x) ~ dx.
    $$
\end{defi}
%%%%%%%%%%%%%%%%%%%%%%%%%%%%%%%%%%%%%%%%%%%%%%%%%%%%
%% Examples
%%%%%%%%%%%%%%%%%%%%%%%%%%%%%%%%%%%%%%%%%%%%%%%%%%%%
\begin{ejer}
    A thin, two-meter bar, represented by the interval $0 \leq x \leq 2$, is made of an alloy whose density in units of kg/m is given by $\rho(x) = 1 + x^2$. What is the mass of the bar?
\end{ejer}
\begin{proof}[Solution]
    The mass of the bar in kilograms is
    \begin{align*}
        m & = \int_{a}^{b} \rho(x) ~ dx \\
        & = \int_{0}^{2} (1 + x^{2}) ~ dx \\
        & = \left. \left( x + \frac{1}{3} x^{3} \right) \right\rvert_{0}^{2} \\
        & = \frac{14}{3}.
    \end{align*}
\end{proof}

\begin{ejer}
    Suppose a force of $10$ N is required to stretch a spring 0.1 m from its equilibrium position and hold it in that position. \textbf{Hint}: Hooke's law states that $F(x) = kx$.
    \begin{enumerate}
        \item Assuming the spring obeys Hooke's law, find the spring constant $k$.
        \item How much work is needed to \textit{compress} the spring 0.5 m from its equilibrium position?
        \item How much work is needed to \textit{stretch} the spring 0.25 m from its equilibrium position?
    \end{enumerate}
\end{ejer}
\begin{proof}[Solution]
    \begin{enumerate}
        \item Since $10 = k (0.1)$, we get $k = 100$ so that $F(x) = 100x$.
        \item The work in joules required to compress the spring from $x=0$ to $x=-0.5$ is
        $$
            W = \int_{a}^{b} F(x) ~ dx = \int_{0}^{-0.5} 100 x ~dx = \left. 50x^{2} \right\rvert_{0}^{-0.5} = 12.5.
        $$
        \item The work in joules required to stretch the spring from $x=0$ to $x=0.25$ is
        $$
            W = \int_{a}^{b} F(x) ~ dx = \int_{0}^{0.25} 100 x ~dx = \left. 50x^{2} \right\rvert_{0}^{0.25} = 3.125.
        $$
    \end{enumerate}
\end{proof}
\end{document}