% \documentclass[compacto,10pt]{aleph-notas}
\documentclass[compacto,10pt,comentarios]{aleph-notas}

% -- Paquetes adicionales
\usepackage{enumitem}
\usepackage{aleph-comandos}
\usepackage{parskip}
\usepackage{graphicx}
\usepackage{xfrac}
\usepackage{tikz}
\usepackage{etoolbox}
\usepackage[framemethod=tikz]{mdframed}
\DeclareFontFamily{U}{skulls}{}
\DeclareFontShape{U}{skulls}{m}{n}{ <-> skull }{}
\newcommand{\skull}{\text{\usefont{U}{skulls}{m}{n}\symbol{'101}}}
\def \ds{\displaystyle}
\def \dfx{\dfrac{d}{dx}}
\DeclareMathOperator{\arccot}{arccot}
\DeclareMathOperator{\arcsec}{arcsec}
\DeclareMathOperator{\arccsc}{arccsc} 
\newcommand*\Heq{\ensuremath{\overset{\kern2pt LH}{=}}}

% -- Datos del libro
\institucion{Southwestern College}
\asignatura{MATH 251: Calculus II}
\tema{Improper Integrals}
\autor{Jesús Pérez Cuarenta}
% \fecha{Fall 2024}

%% --> Logos de las guias
\logouno[4.5cm]{../Images/swc_logo}
\definecolor{colordef}{cmyk}{0.81, 0.62, 0.00, 0.22}

%% -- Solucion para alumnos
% \AtBeginEnvironment{proof}{\color{white}}

%% 
% https://www.emathhelp.net/en/calculators/algebra-2/partial-fraction-decomposition-calculator/

\begin{document}

\encabezado

\section*{Improper Integrals}
\begin{mdframed}
    \center Learning Objectives \\
    \begin{itemize}
        \item Define Type I and Type II improper integrals.
        \item Evaluate Type I and Type II improper integrals.
    \end{itemize}
\end{mdframed}

\begin{defi}[Improper Integrals over Infinite Intervals (Type 1)]
    \begin{enumerate}
        \item If $f$ is continuous on $[a, \infty)$, then
        $$
            \int_{a}^{\infty} f(x) ~ dx = \lim_{b \to \infty} \int_{a}^{b} f(x) ~ dx.
        $$
        \item If $f$ is continuous on $(-\infty, b]$, then
        $$
            \int_{-\infty}^{b} f(x) ~ dx = \lim_{a \to -\infty} \int_{a}^{b} f(x) ~ dx .
        $$
        \item If $f$ is continuous on $(-\infty, \infty)$, then
        $$
            \int_{-\infty}^{\infty} f(x) ~ dx = \lim_{a \to -\infty} \int_{a}^{c} f(x) ~ dx + \lim_{b \to \infty} \int_{c}^{b} f(x) ~ dx 
        $$
        where $c$ is any real number.
    \end{enumerate}
    If the limits in cases 1-3 exist, then the improper integrals \textbf{converge}; otherwise, they \textbf{diverge}.
\end{defi}
`   1'
\begin{ejer}
    Evaluate the improper integral
    $$
        \int_{1}^{\infty} \frac{1}{x^{2}} ~ dx ~ .
    $$
\end{ejer}
\begin{proof}[Solution]
    We have
    \begin{align*}
        \int_{1}^{\infty} \frac{1}{x^{2}} ~ dx
            & = \lim_{b \to \infty} \int_{1}^{b} \frac{1}{x^{2}}~dx \\
            & = \lim_{b \to \infty} \left. -\frac{1}{x} \right\rvert_{1}^{b} \\
            & = \lim_{b \to \infty} \left( -\frac{1}{(b)} + \frac{1}{(1)} \right) \\
            & = 1 ~ .
    \end{align*}
    Since the limit exists, the integral is \textbf{convergent}.
\end{proof}

\begin{ejer}
    Evaluate the improper integral
    $$
        \int_{1}^{\infty} \frac{1}{\sqrt{x}} ~ dx ~ .
    $$
\end{ejer}
\begin{proof}[Solution]
    We have
    \begin{align*}
        \int_{1}^{\infty} \frac{1}{\sqrt{x}} ~ dx
            & = \lim_{b \to \infty} \int_{1}^{b} \frac{1}{\sqrt{x}} ~ dx \\
            & = \lim_{b \to \infty} \left. 2\sqrt{x} \right\rvert_{1}^{b} \\
            & = \lim_{b \to \infty} 2 \left( \sqrt{b} - \sqrt{1} \right) \\
            & = \infty ~ .
    \end{align*}
    Since the limit does not exist, the integral is \textbf{divergent}.
\end{proof}

\begin{ejer}
    Evaluate the improper integral
    $$
        \int_{-\infty}^{-3} \frac{1}{x^{4}} ~ dx ~ .
    $$
\end{ejer}
\begin{proof}[Solution]
    We have
    \begin{align*}
        \int_{-\infty}^{-3} \frac{1}{x^{4}} ~ dx
            & = \lim_{a \to -\infty} \int_{a}^{-3} \frac{1}{x^{4}} ~ dx \\
            & = \lim_{a \to -\infty} \left. -\frac{1}{3x^{3}} \right\rvert_{a}^{-3} \\
            & = \lim_{a \to -\infty} -\frac{1}{3} \left( \frac{1}{(-3)^{3}} - \frac{1}{(a)^{3}} \right) \\
            & = \frac{1}{81} ~ .
    \end{align*}
    Since the limit exists, the integral is \textbf{convergent}.
\end{proof}

\begin{ejer}
    Evaluate the improper integral
    $$
        \int_{-\infty}^{4} 8\mathrm{e}^{(-5x)} ~ dx ~ .
    $$
\end{ejer}
\begin{proof}[Solution]
    We have
    \begin{align*}
        \int_{-\infty}^{4} 8\mathrm{e}^{(-5x)} ~ dx
            & = \lim_{a \to -\infty} \int_{a}^{4} 8\mathrm{e}^{(-5x)} ~ dx \\
            & = \lim_{a \to -\infty} \left. -\frac{8}{5}  \mathrm{e}^{(-5x)} \right\rvert_{a}^{4} \\
            & = \lim_{a \to -\infty} -\frac{8}{5} \left( \mathrm{e}^{(-5(4))} - \mathrm{e}^{(-5(a))} \right) \\
            & = \infty ~ .
    \end{align*}
    Since the limit does not exist, the integral is \textbf{divergent}.
\end{proof}

\begin{ejer}
    Evaluate the improper integral
    $$
        \int_{-\infty}^{\infty} \frac{1}{1 + x^{2}} ~ dx ~ .
    $$
\end{ejer}
\begin{proof}[Solution]
    We have
    \begin{align*}
        \int_{-\infty}^{\infty} \frac{1}{1 + x^{2}} ~ dx
            & = \lim_{a \to -\infty} \int_{a}^{0} \frac{1}{1 + x^{2}} ~ dx 
                + \lim_{b \to \infty} \int_{0}^{b} \frac{1}{1 + x^{2}} ~ dx  \\
            & = \lim_{a \to -\infty} \left. \arctan(x) \right\rvert_{a}^{0}
                + \lim_{b \to \infty} \left. \arctan(x) \right\rvert_{0}^{b} \\
            & = \lim_{a \to -\infty} \left( \arctan(0) - \arctan(a) \right)
                + \lim_{b \to \infty} \left( \arctan(b) - \arctan(0) \right) \\
            & = \frac{\pi}{2} + \frac{\pi}{2} \\
            & = \pi ~ .
    \end{align*}
    Since the limit exists, the integral is \textbf{convergent}.
\end{proof}

\begin{teo}[Improper Integrals over Infinite Intervals (Type 2)]
    \begin{enumerate}
        \item Suppose $f$ is continuous on $(a, b]$ with
        $$
            \lim_{x \to a^{+}} f(x) = \pm \infty.
        $$
        Then,
        $$
            \int_{a}^{b} f(x) ~ dx = \lim_{c \to a^{+}} \int_{c}^{b} f(x) ~ dx .
        $$
        \item Suppose $f$ is continuous on $[a, b)$ with
        $$
            \lim_{x \to b^{-}} f(x) = \pm \infty.
        $$
        Then,
        $$
            \int_{a}^{b} f(x) ~ dx = \lim_{c \to b^{-}} \int_{a}^{c} f(x) ~ dx .
        $$
        \item Suppose $f$ is continuous on $[a, b]$ except at the interior point $p$ where $f$ is unbounded. Then
        $$
            \int_{a}^{b} f(x) ~ dx = \lim_{c \to p^{-}} \int_{a}^{c} f(x) ~ dx 
                + \lim_{d \to p^{+}} \int_{d}^{b} f(x) ~ dx 
        $$
    \end{enumerate}
    If the limits in cases 1-3 exist, then the improper integrals \textbf{converge}; otherwise, they \textbf{diverge}.
\end{teo}

\begin{ejer}
    Evaluate the improper integral
    $$
        \int_{1}^{4} \frac{1}{\sqrt{x - 1}} ~ dx ~ .
    $$
\end{ejer}
\begin{proof}[Solution]
    Note the integrand is discontinuous at $x = 1$. We introduce a limit in order to arrive at a solution,
    \begin{align*}
        \int_{1}^{4} \frac{1}{\sqrt{x - 1}} ~ dx 
            & = \lim_{a \to 1^{+}} \int_{a}^{4} \frac{1}{\sqrt{x - 1}} ~ dx \\
            & = \lim_{a \to 1^{+}} \left. 2 \sqrt{x - 1} \right\rvert_{a}^{4} \\
            & = \lim_{a \to 1^{+}} 2 \left( \sqrt{4 - 1} - \sqrt{a - 1} \right) \\
            & = 2 \left( \sqrt{4 - 1} - \sqrt{1^{+} - 1} \right) \\
            & = 2 \left( \sqrt{3} - \sqrt{0^{+}} \right) \\
            & = 2\sqrt{3} ~ .
    \end{align*}
    Since the limit exists, the integral is \textbf{convergent}.
\end{proof}

\begin{ejer} $\skull$
    Evaluate the improper integral
    $$
        \int_{0}^{5} \frac{1}{x - 3} ~ dx ~ .
    $$
\end{ejer}
\begin{proof}[Solution] 
    Note the integrand is discontinuous at $x = 3$. We introduce limits in order to arrive at a solution,
    \begin{align*}
        \int_{0}^{5} \frac{1}{x - 3} ~ dx
            & = \lim_{b \to 3^{-}} \int_{0}^{b} \frac{1}{x - 3} ~ dx 
                + \lim_{a \to 3^{+}} \int_{a}^{5} \frac{1}{x - 3} ~ dx \\
            & = \lim_{b \to 3^{-}} \left( \ln(|x - 3|) \right\rvert_{0}^{b} 
                + \lim_{a \to 3^{+}} \left. \ln(|x - 3|) \right\rvert_{a}^{5} \\ 
            & = \lim_{b \to 3^{-}} \left( \ln(|b - 3|) - \ln(|0 - 3|) \right)
                + \lim_{a \to 3^{+}} \left( \ln(|5 - 3|) - \ln(|a - 3|) \right) \\
            & = \left( \ln(|3^{-} - 3|) - \ln(3) \right)
                +  \left( \ln(2) - \ln(|3^{+} - 3|) \right) \\
            & = \left( \ln(0^{+}) - \ln(3) \right)
                +  \left( \ln(2) - \ln(0^{+}) \right) ~ .
    \end{align*}
    Since at least one integral does not converge (e.g., for $x < 3$), we conclude that
    $$
        \int_{0}^{5} \frac{1}{x - 3} ~ dx
    $$
    is \textbf{divergent}.
\end{proof}

The last example is interesting. Can you spot why? If you need some motivation, solve
$$
    \int_{-1}^{1} \frac{1}{x} ~ dx = \lim_{\epsilon \to 0^{+}} \left( \int_{-1}^{-\epsilon} \frac{1}{x} ~ dx + \int_{\epsilon}^{1} \frac{1}{x} ~ dx \right)
$$
and
$$
    \int_{-1}^{1} \frac{1}{x} ~ dx = \lim_{\epsilon \to 0^{+}} \left( \int_{-1}^{-\epsilon} \frac{1}{x} ~ dx + \int_{2\epsilon}^{1} \frac{1}{x} ~ dx \right) ~ .
$$ 
\end{document}