% \documentclass[compacto,10pt]{aleph-notas}
\documentclass[compacto,10pt,comentarios]{aleph-notas}

% -- Paquetes adicionales
\usepackage{enumitem}
\usepackage{aleph-comandos}
\usepackage{parskip}
\usepackage{graphicx}
\usepackage{xfrac}
\usepackage{tikz}
\usepackage{etoolbox}
\usepackage[framemethod=tikz]{mdframed}
\DeclareFontFamily{U}{skulls}{}
\DeclareFontShape{U}{skulls}{m}{n}{ <-> skull }{}
\newcommand{\skull}{\text{\usefont{U}{skulls}{m}{n}\symbol{'101}}}
\def \ds{\displaystyle}
\def \dfx{\dfrac{d}{dx}}
\DeclareMathOperator{\arccot}{arccot}
\DeclareMathOperator{\arcsec}{arcsec}
\DeclareMathOperator{\arccsc}{arccsc}
\newcommand*\Heq{\ensuremath{\overset{\kern2pt LH}{=}}}

% -- Datos del libro
\institucion{Southwestern College}
\asignatura{MATH 251: Calculus II}
\tema{Trigonometric Integrals}
\autor{Jesús Pérez Cuarenta}
% \fecha{Fall 2024}

%% --> Logos de las guias
\logouno[4.5cm]{../Images/swc_logo}
\definecolor{colordef}{cmyk}{0.81, 0.62, 0.00, 0.22}

%% -- Solucion para alumnos
% \AtBeginEnvironment{proof}{\color{white}}

\begin{document}

\encabezado

\section*{Trigonometric Integrals}
\begin{mdframed}
    \center Learning Objectives \\
    \begin{itemize}
        \item Evaluate integrals involving trigonometric functions.
    \end{itemize}
\end{mdframed}

This section mainly covers strategies surrounding integrals and trigonometric functions.

\subsection*{Strategy to Evaluate $\int \sin^{m}(x)\cos^{n}(x) ~ dx$}
\begin{itemize}
    \item The variable $n$ is odd and positive, and $m$ is real

    Leave one factor of $\cos(x)$ alone and rewrite the remaining factors in terms of the sine function via
    $$
        \cos^{2}(x) = 1 - \sin^{2}(x) ~.
    $$
    \item The variable $m$ is odd and positive, $n$ is real

    Leave one factor of $\sin(x)$ alone and rewrite the remaining factors in terms of the cosine function via
    $$
        \sin^{2}(x) = 1 - \cos^{2}(x) ~ .
    $$
    \item The powers of both $\sin(x)$ and $\cos(x)$ are both even and positive integers

    Use half-angle formulas to transform the integrand into a polynomial in $\cos(2x)$, and apply the preceding strategies once again to powers of $\cos(2x)$ greater than 1.
\end{itemize}

\begin{ejer}
    Evaluate the following integral
    $$
        \int \cos^{3} \left( \frac{x}{3} \right) ~ dx ~ .
    $$
\end{ejer}
\begin{proof}[Solution]
    We have
    \begin{align*}
        \int \cos^{3} \left( \frac{x}{3} \right) ~ dx
        & = \int \cos \left( \frac{x}{3} \right) \cos^{2} \left( \frac{x}{3} \right) ~ dx \\
        & = \int \cos \left( \frac{x}{3} \right) \left( 1 - \sin^{2} \left( \frac{x}{3} \right) \right) ~ dx ~ .
    \end{align*}
    Let $u = \sin(\frac{x}{3})$ and with $du = \frac{1}{3} \cos(\frac{x}{3}) ~ dx$ we get
    \begin{align*}
        \int \cos \left( \frac{x}{3} \right) \left( 1 - \sin^{2} \left( \frac{x}{3} \right) \right) ~ dx
        & = 3 \int  1 - u^{2} ~ du \\
        & = 3 \left( u - \frac{1}{3}u^{3} \right) + C \\
        & = 3 \sin\left(\frac{x}{3}\right) - \sin^{3}\left(\frac{x}{3}\right) + C ~ .
    \end{align*}
\end{proof}

\begin{ejer}
    Evaluate the following integral
    $$
        \int \sin^{4}(x) ~ dx ~ .
    $$
\end{ejer}
\begin{proof}[Solution]
    We have
    \begin{align*}
        \int \sin^{4}(x) ~ dx
        & = \int \left( \sin^{2}(x) \right)^{2} ~ dx \\
        & = \int \left( \frac{1}{2} \left( 1 - \cos(2x) \right) \right)^{2} ~ dx \\
        & = \frac{1}{4} \int 1 - 2 \cos(2x) + \cos^{2}(2x) ~ dx \\
        & = \frac{1}{4} \int 1 - 2 \cos(2x) + \frac{1}{2}(1+\cos(4x)) ~ dx \\ 
        & = \frac{1}{4} \int \frac{3}{2} - 2 \cos(2x) + \frac{1}{2}\cos(4x) ~ dx \\
        & = \frac{1}{4} \left( \int \frac{3}{2} ~ dx -2 \int \cos(2x) ~ dx + \frac{1}{2} \int \cos(4x) ~ dx \right) \\
        & = \frac{1}{4} \left( \frac{3}{2} x - \sin(2x) + \frac{1}{8} \sin(4x) \right) + C \\
        & = \frac{3}{8} x - \frac{1}{4} \sin(2x) + \frac{1}{32} \sin(4x) + C ~ .
    \end{align*}
\end{proof}

\begin{ejer}
    Evaluate the following integral
    $$
        \int \sin^{5}(x) \cos^{2}(x) ~ dx ~ .
    $$
\end{ejer}
\begin{proof}[Solution]
    We have
    \begin{align*}
        \int \sin^{5}(x) \cos^{2}(x) ~ dx 
        & = \int \sin(x) \sin^{4}(x) \cos^{2}(x) ~ dx \\
        & = \int \sin(x) (\sin^{2}(x))^{2} \cos^{2}(x) ~ dx \\
        & = \int \sin(x) (1 - \cos^{2}(x))^{2} \cos^{2}(x) ~ dx \\
        & = \int \sin(x) (1 - 2\cos^{2}(x) + \cos^{4}(x)) \cos^{2}(x) ~ dx \\
        & = \int \sin(x) (\cos^{2}(x) - 2\cos^{4}(x) + \cos^{6}(x)) ~ dx \\
        & = - \int u^{2} - 2 u^{4} + u^{6} ~ du \tag*{by letting $u = \cos(x)$} \\
        & = - \frac{1}{3} u ^{3} + \frac{2}{5} u^{5} - \frac{1}{7}u^{7} + C \\
        & = - \frac{1}{3} \cos^{3}(x) + \frac{2}{5} \cos^{5}(x) - \frac{1}{7}\cos^{7}(x) + C ~ .
    \end{align*}
\end{proof}

\begin{ejer}
    Evaluate the following integral
    $$
        \int \cos^{3}(x)\sqrt{\sin^{3}(x)} ~ dx ~ .
    $$
\end{ejer}
\begin{proof}[Solution]
    We have
    \begin{align*}
        \int \cos^{3}(x)\sqrt{\sin^{3}(x)} 
        & = \int \cos(x) \cos^{2}(x) \sqrt{\sin^{3}(x)} ~ dx \\
        & = \int \cos(x) \left( 1 - \sin^{2}(x) \right) \sin^{\frac{3}{2}}(x) ~ dx \\
        & = \int \cos(x) \left( \sin^{\frac{3}{2}}(x) - \sin^{\frac{7}{2}}(x) \right) ~ dx \\
        & = \int u^{\frac{3}{2}} - u^{\frac{7}{2}} ~ du \tag*{by letting $u = \sin(x)$} \\
        & = \frac{2}{5} u ^{\frac{5}{2}} - \frac{2}{9} u ^{\frac{9}{2}} \\
        & = \frac{2}{5} \sin^{\frac{5}{2}}(x) - \frac{2}{9}  \sin^{\frac{9}{2}}(x) + C ~ .
    \end{align*}
\end{proof}

\subsection*{Using Product Identities}
Integrals of the form
$$
    \int \sin(mx) \cos(nx) ~ dx, \int \sin(mx) \sin(nx) ~ dx, \int \cos(mx)\cos(nx) ~dx
$$
require the use of
\begin{align*}
    \sin(A)\cos(B) = \frac{1}{2} \left( \sin(A + B) + \sin(A - B) \right), \\
    \sin(A)\sin(B) = \frac{1}{2} \left( \cos(A - B) - \cos(A + B) \right), \\ 
    \cos(A)\cos(B) = \frac{1}{2} \left( \cos(A - B) + \cos(A +B)\right).
\end{align*}

\begin{ejer}
    Evaluate
    $$
        \int \sin(6\theta) \cos(13\theta)~ d\theta.
    $$
\end{ejer}
\begin{proof}
    We have
    \begin{align*}
        \int \sin(6\theta) \cos(13\theta)~ d\theta
        & = \frac{1}{2} \int \sin\left( 6\theta + 13\theta \right) + \sin\left( 6\theta - 13\theta \right) ~ d\theta \\
        & = \frac{1}{2} \int \sin\left( 19\theta \right)  + \sin\left( -7 \theta \right)  ~ d\theta \\
        & = \frac{1}{2} \int \sin\left( 19\theta \right)  - \sin\left(7 \theta \right)  ~ d\theta \tag*{since $\sin(x)$ is odd} \\
        & = \frac{1}{2} \left( -\frac{1}{19} \cos(19\theta) + \frac{1}{7} \cos\left(7 \theta \right) \right)  + C  \\
        & = -\frac{1}{38} \cos(19\theta) + \frac{1}{14} \cos\left(7 \theta \right)  + C ~ .
    \end{align*}
\end{proof}
\end{document}