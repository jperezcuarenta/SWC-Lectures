% \documentclass[compacto,10pt]{aleph-notas}
\documentclass[compacto,10pt,comentarios]{aleph-notas}

% -- Paquetes adicionales
\usepackage{enumitem}
\usepackage{aleph-comandos}
\usepackage{parskip}
\usepackage{graphicx}
\usepackage{xfrac}
\usepackage{tikz}
\usepackage{etoolbox}
\usetikzlibrary{positioning}
\usepackage[framemethod=tikz]{mdframed}
\DeclareFontFamily{U}{skulls}{}
\DeclareFontShape{U}{skulls}{m}{n}{ <-> skull }{}
\newcommand{\skull}{\text{\usefont{U}{skulls}{m}{n}\symbol{'101}}}
\def \ds{\displaystyle}
\def \dfx{\dfrac{d}{dx}}
\DeclareMathOperator{\arccot}{arccot}
\DeclareMathOperator{\arcsec}{arcsec}
\DeclareMathOperator{\arccsc}{arccsc}
\newcommand*\Heq{\ensuremath{\overset{\kern2pt LH}{=}}}

% -- Datos del libro
\institucion{Southwestern College}
\asignatura{MATH 251: Calculus II}
\tema{Integration By Parts}
\autor{Jesús Pérez Cuarenta}
% \fecha{Fall 2024}

%% --> Logos de las guias
\logouno[4.5cm]{../Images/swc_logo}
\definecolor{colordef}{cmyk}{0.81, 0.62, 0.00, 0.22}

%% -- Solucion para alumnos
% \AtBeginEnvironment{proof}{\color{white}}

\begin{document}

\encabezado

\section*{Integration By Parts}
\begin{mdframed}
    \center Learning Objectives \\
    \begin{itemize}
        \item Use integration by parts, when appropriate, to evaluate integrals.
        \item Evaluate integrals requiring repeated integration by parts.
        \item Solve application problems requiring integration by parts.
    \end{itemize}
\end{mdframed}

In this section we cover a popular integration technique. The idea is that we can represent a difficult integral in terms of a simpler one. We start with a proof and then cover some examples.
\begin{teo}[Integration by Parts]
    Let $u(x), u'(x), v(x), v'(x)$ be continuous functions, then
    $$
        \int u(x) v'(x) ~ dx = u(x) v(x)  - \int u'(x) v(x) ~ dx.
    $$
\end{teo}
\begin{proof}[Proof]
    From the Product Rule, we get
    \begin{align*}
        \frac{d}{dx} \left( u(x)v(x) \right)
            & = u(x) \left(\frac{d}{dx} v(x) \right) + \left( \frac{d}{dx} u(x) \right) v(x) \\
        \implies
        \int \left( \frac{d}{dx} u(x)v(x) \right) dx
        & = \int \left( u(x) \left(\frac{d}{dx} v(x) \right) + \left( \frac{d}{dx} u(x) \right) v(x) \right) dx \\
        & = \int u(x) \left(\frac{d}{dx} v(x) \right) dx + \int \left( \frac{d}{dx} u(x) \right) v(x) ~ dx \\
        & = \int u(x) ~ dv + \int v(x) ~ du,
    \end{align*}
    $\therefore$
    $$
        \int u(x) v'(x) ~ dx = u(x) v(x)  - \int u'(x) v(x) ~ dx.
    $$
\end{proof}
We will commonly use the shorthand notation
$$
    \int u ~ dv = uv - \int v ~ du . 
$$
The acronym \textbf{LIATE} is helpful for determining which function should be denoted by $u$. It is a way to memorize a hierarchy of functions (logarithmic, inverse trig, algebraic, trigonometric, and exponential). Please note there is no guarantee this approach always works.

\begin{ejer}
    Use integration by parts to evaluate the integral.
    $$
        \int x \mathrm{e}^{(3x)} ~ dx
    $$
\end{ejer}
\begin{proof}[Solution]
    Let $u = x$ and $dv = \mathrm{e}^{(3x)} ~ dx$. We note that
    \begin{align*}
        u & = x \\ 
        \implies du & = dx
    \end{align*}
    and
    \begin{align*}     
        v & = \frac{1}{3} \mathrm{e}^{(3x)} \\ 
        \implies dv & = \mathrm{e}^{(3x)} ~ dx.
    \end{align*}
    Thus,
    \begin{align*}
        \int x \mathrm{e}^{(3x)} ~ dx 
        & = \int u ~ dv \\
        & =  uv - \int v ~ du \\
        & = \frac{1}{3}x\mathrm{e}^{(3x)} - \frac{1}{3} \int \mathrm{e}^{(3x)} ~ dx \\ 
        & = \frac{1}{3}x\mathrm{e}^{(3x)} - \frac{1}{9} \mathrm{e}^{(3x)} + C .
    \end{align*}
\end{proof}

\begin{ejer}
    Use integration by parts to evaluate the integral.
    $$
        \int \frac{\ln(x)}{x^{2}} ~ dx
    $$
    Assume $x > 0$.
\end{ejer}
\begin{proof}[Solution]
    Let $u = \ln(x)$ and $dv = \frac{1}{x^{2}} ~ dx$. We note that
    \begin{align*}
        u & = \ln(x) \\ 
        \implies du & = \frac{1}{x} ~ dx
    \end{align*}
    and
    \begin{align*}     
        v & = -\frac{1}{x} \\ 
        \implies dv & = \frac{1}{x^{2}} ~ dx.
    \end{align*}
    Thus,
    \begin{align*}
        \int \frac{\ln(x)}{x^{2}} ~ dx
        & = -\frac{\ln(x)}{x} + \int \frac{1}{x^2}~dx \\
        & =  -\frac{\ln(x)}{x} - \frac{1}{x} + C .
    \end{align*}
\end{proof}

\begin{ejer}
    Use integration by parts to evaluate the integral.
    $$
        \int \arcsin(x) ~ dx
    $$
\end{ejer}
\begin{proof}[Solution]
    Let $u = \arcsin(x)$ and $dv = dx$. We note that
    \begin{align*}
        u & = \arcsin(x) \\ 
        \implies du & = \frac{1}{\sqrt{1-x^{2}}} ~ dx
    \end{align*}
    and
    \begin{align*}     
        v & = x \\ 
        \implies dv & = dx.
    \end{align*}
    Thus,
    \begin{align*}
        \int \arcsin(x) ~ dx
        & = x \arcsin(x) - \int \frac{x}{\sqrt{1 - x^{2}}}~dx \\
        & =  x \arcsin(x) + \sqrt{1 - x^{2}} + C .
    \end{align*}
\end{proof}

% \begin{ejer}
%     Use integration by parts to evaluate the integral.
%     $$
%         \int \cos(\sqrt{5x}) ~ dx
%     $$
% \end{ejer}
% \begin{proof}[Solution]
%     Take $ u = \sqrt{5x}$. Then $du = \sqrt{5x} ~ dx$ and hence
%     $$
%         \int \cos(\sqrt{5x})~dx = \frac{\sqrt{5}}{5} \int \cos(u)~du
%     $$
%     % Let $u = \cos(\sqrt{5x})$ and $dv = dx$. We note that
%     % \begin{align*}
%     %     u & = \cos(\sqrt{5x}) \\ 
%     %     \implies du & = -\frac{\sqrt{5}}{2}\frac{\sin(\sqrt{5x})}{\sqrt{x}} ~ dx
%     % \end{align*}
%     % and
%     % \begin{align*}     
%     %     v & = x \\ 
%     %     \implies dv & = dx.
%     % \end{align*}
%     % Thus,
%     % \begin{align*}
%         % \int \cos(\sqrt{5x}) ~ dx
%         % & = x \cos(\sqrt{5x}) + \frac{\sqrt{5}}{2} \int x \frac{\sin(\sqrt{5x})}{\sqrt{x}} ~ dx \\
%         % & =  x \cos(\sqrt{5x}) + \frac{\sqrt{5}}{2} \int \sqrt{x} \sin(\sqrt{5x}) \\
%         % & = 1
%     % \end{align*}
% \end{proof}

\begin{ejer}
    Use integration by parts to evaluate the integral.
    $$
        \int \mathrm{e}^{\sqrt{x}} ~ dx
    $$
\end{ejer}
\begin{proof}[Solution]
    Let $u = \sqrt{x}$, then $du = \frac{1}{2\sqrt{x}}~dx$. Notice
    $$
    du = \frac{1}{2\sqrt{x}}~dx \implies 2u~du = dx
    $$
    Now,
    \begin{align*}
        \int \mathrm{e}^{\sqrt{x}} ~ dx
            & = 2 \int u \mathrm{e}^{u} ~ du \\
            & = 2 u\mathrm{e}^{u} - 2 \mathrm{e}^{u} + C \\
            & = 2 \sqrt{x}\mathrm{e}^{\sqrt{x}} - 2 \mathrm{e}^{\sqrt{x}} + C .
    \end{align*}
\end{proof}

% \begin{ejer}
%     Compare
%     $$
%         \int x \sin(x) ~ dx
%     $$
%     with
%     $$
%         \int x \sin^{2}(x) ~ dx .
%     $$
% \end{ejer}
% \begin{proof}[Solution]
%     Test
% \end{proof}

%%%%
% Add motiviation for repeated integration by parts, or just a statement.
%%%%

\begin{ejer}
    Use repeated integration by parts to evaluate the integral
    $$
        \int x^{2} \sin(2x) ~ dx ~ .
    $$
\end{ejer}
\begin{proof}[Solution]
    Let $u = x^{2}$ and $dv = \sin(2x)~dx$. We have
    \begin{align*}
        \int x^{2} \sin(2x) ~ dx
        & = -\frac{1}{2}x^{2}\cos(2x) + \int x \cos(2x)~dx .
    \end{align*}
    Now, let $ \hat{u} = x$ and $d\hat{v} = \cos(2x)~dx$ so that
    \begin{align*}
        \int x \cos(2x)~dx 
            & = \frac{1}{2}x\sin(2x) - \frac{1}{2} \int \sin(2x)~dx \\
            & = \frac{1}{2}x\sin(2x) + \frac{1}{4} \cos(2x) .
    \end{align*}
    Finally,
    $$
    \int x^{2} \sin(2x) ~ dx 
    = -\frac{1}{2}x^{2}\cos(2x) + \frac{1}{2}x\sin(2x) + \frac{1}{4} \cos(2x).
    $$
\end{proof}

%%%
% Add tabular method, "there's a pattern..." and so on.
%%%

\begin{ejer}
    Use the tabular method to evaluate the integral
    $$
        \int x^{3} \sin(2x) ~ dx ~ .
    $$
\end{ejer}
\begin{proof}[Solution]
    We let $u = x^{3}$ and $dv = \sin(2x) ~ dx$. With the tabular method we draw a table with two columns containing derivatives and integrals. We write derivatives of $u$ on the left column. In addition, we integrate $dv$ on the right column.

    \begin{center}
    \begin{tikzpicture}
        % first column
        \node (a) at (0,0) {$x^3$};
        \node[below=5mm of a] (b) {$3x^{2}$};
        \node[below=5mm of b] (c) {$6x$};
        \node[below=5mm of c] (d) {$6$};
        \node[below=5mm of d] (e) {$0$};
        % second column
        \node[right=1cm of a] (a1)  {$\sin(2x)$};
        \node[below=6mm of a1] (b1) {$-\frac{1}{2}\cos(2x)$};
        \node[below=5mm of b1] (c1) {$-\frac{1}{4}\sin(2x)$};
        \node[below=5mm of c1] (d1) {$\frac{1}{8}\cos(2x)$};
        \node[below=5mm of d1] (e1) {$\frac{1}{16}\sin(2x)$};
        % arrows
        \draw[->,blue!70] (a.320) -- (b1.west) node[black,pos=.65, above] {$+$};
        \draw[->,blue!70] (b.320) -- (c1.west) node[black,pos=.65, above] {$-$};
        \draw[->,blue!70] (c.320) -- (d1.west) node[black,pos=.65, above] {$+$};
        \draw[->,blue!70] (d.320) -- (e1.west) node[black,pos=.65, above] {$-$};
    \end{tikzpicture}
    \end{center}

    Using the table, we arrive at the answer by inspection. Namely,
    $$
        \int x^{3} \sin(2x) ~ dx
        = -\frac{1}{2} x^{3} \cos(2x) + \frac{3}{4} x^{2} \sin(2x) + \frac{3}{4} x \cos(2x) - \frac{3}{8}\sin(2x)
    $$
    and we are done.
\end{proof}

%%%
% Integrals of the form $e^{ax} \sin(bx)$ or $e^{ax} \cos(bx)$. 
%%%

\begin{ejer}
    Evaluate
    $$
        \int \mathrm{e}^{(2x)} \sin(x) ~ dx ~ .
    $$
\end{ejer}
\begin{proof}[Solution]
    Let $u = \sin(x)$ and $dv = \mathrm{e}^{(2x)} ~ dx$. We have
    \begin{align*}
        \int \mathrm{e}^{(2x)} \sin(x) ~ dx
        & = \frac{1}{2}\mathrm{e}^{(2x)}\sin(x) - \frac{1}{2} \left( \int \mathrm{e}^{(2x)} \cos(x) ~ dx \right) .
    \end{align*}
    Now take $\hat{u} = \cos(x)$ and $d\hat{v} = \mathrm{e}^{(2x)} ~ dx$ so that we get
    \begin{align*}
        \int \mathrm{e}^{(2x)} \cos(x) ~ dx 
        & = \frac{1}{2}\mathrm{e}^{(2x)}\cos(x) + \frac{1}{2} \int \mathrm{e}^{(2x)}\sin(x) ~ dx ~ .
    \end{align*}
    Let
    $$
        I = \int \mathrm{e}^{(2x)} \sin(x) ~ dx
    $$
    which implies
    \begin{align*}
        I & = \frac{1}{2}\mathrm{e}^{(2x)}\sin(x) -\frac{1}{2} \left( \frac{1}{2}\mathrm{e}^{(2x)}\cos(x) + \frac{1}{2} I  \right) \\
        \implies I & = \frac{1}{2}\mathrm{e}^{(2x)}\sin(x) - \frac{1}{4}\mathrm{e}^{(2x)}\cos(x) - \frac{1}{4} I  \\
        \implies \left( 1 + \frac{1}{4} \right) I & = \frac{1}{2}\mathrm{e}^{(2x)}\sin(x) - \frac{1}{4}\mathrm{e}^{(2x)}\cos(x) \\
        \implies I & = \frac{4}{5} \left( \frac{1}{2}\mathrm{e}^{(2x)}\sin(x) - \frac{1}{4}\mathrm{e}^{(2x)}\cos(x) \right)
    \end{align*}
    We've ignored the constant of integration thus far but include it in our final answer. Finally,
    $$
    \int \mathrm{e}^{(2x)} \sin(x) ~ dx = \frac{2}{5}\mathrm{e}^{(2x)}\sin(x) - \frac{1}{5}\mathrm{e}^{(2x)}\cos(x) + C .
    $$
\end{proof}

\begin{ejer}
    Evaluate the definite integral
    $$
        \int_{0}^{\frac{\pi}{4}} x \sin(2x) ~ dx ~ .
    $$
\end{ejer}
\begin{proof}[Solution]
    Integration by parts shows
    $$
        \int_{0}^{\frac{\pi}{4}} x \sin(2x) ~ dx
        = -\frac{1}{2}x\cos(2x) + \frac{1}{4} \sin(2x) + C.
    $$
    Since we are dealing with a definite integral, we have
    \begin{align*}
        \int_{0}^{\frac{\pi}{4}} x \sin(2x) ~ dx
        & = \left. -\frac{1}{2}x\cos(2x) + \frac{1}{4} \sin(2x) \right\rvert_{0}^{\frac{\pi}{4}} \\
        & = \frac{1}{4} ~ .
        % & = \left( -\frac{1}{2}\frac{\pi}{4}\cos(2\frac{\pi}{4}) + \frac{1}{4} \sin(2\frac{\pi}{4}) \right) - \left( -\frac{1}{2}\cdot 0 \cdot \cos(2 \cdot x) + \frac{1}{4} \sin(2 \cdot 0) \right)
    \end{align*}
\end{proof}

%%%
% Applications
%%%
\begin{ejer}
    A drug taken orally is absorbed into the bloodstream at the rate of $t\mathrm{e}^{(-\frac{2}{5}t)}$ milligrams per hour,
    where t is the number of hours since the drug was taken. Find the total amount of drug
    absorbed during the first 6 hours.
\end{ejer}
\begin{proof}[Solution]
    Define $h'(t) = t\mathrm{e}^{(-\frac{2}{5}t)}$, we wish to obtain $h(t)$ so that we may calculate $h(6) - h(0)$ to find the total amount of drug absorbed during the first six hours. Integrating $h'(t)$ produces
    \begin{align*}
        \int_{0}^{6} h'(t) dt & = \int_{0}^{6} t\mathrm{e}^{(-\frac{2}{5}t)} ~ dt \\
        & = \left. -\frac{5}{2}t\mathrm{e}^{(-\frac{2}{5}t)} - \frac{25}{4} \mathrm{e}^{(-\frac{2}{5}t)} \right\rvert_{0}^{6} \\
        & = -\frac{85}{4}\mathrm{e}^{(-\frac{12}{5})} + \frac{25}{4} ~ .
    \end{align*}
    We take an approximation rounded to two decimal places to produce a sensible result. We have that $4.32$ milligrams are absorbed during the first six hours after taking the drug.
\end{proof}
\end{document}