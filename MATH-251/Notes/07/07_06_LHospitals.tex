% \documentclass[compacto,10pt]{aleph-notas}
\documentclass[compacto,10pt,comentarios]{aleph-notas}

% -- Paquetes adicionales
\usepackage{enumitem}
\usepackage{aleph-comandos}
\usepackage{parskip}
\usepackage{graphicx}
\usepackage{xfrac}
\usepackage{tikz}
\usepackage{etoolbox}
\usepackage[framemethod=tikz]{mdframed}
\DeclareFontFamily{U}{skulls}{}
\DeclareFontShape{U}{skulls}{m}{n}{ <-> skull }{}
\newcommand{\skull}{\text{\usefont{U}{skulls}{m}{n}\symbol{'101}}}
\def \ds{\displaystyle}
\def \dfx{\dfrac{d}{dx}}
\DeclareMathOperator{\arccot}{arccot}
\DeclareMathOperator{\arcsec}{arcsec}
\DeclareMathOperator{\arccsc}{arccsc}
\newcommand*\Heq{\ensuremath{\overset{\kern2pt LH}{=}}}

% -- Datos del libro
\institucion{Southwestern College}
\asignatura{MATH 251: Calculus II}
\tema{L’Hôpital's Rule}
\autor{Jesús Pérez Cuarenta}
% \fecha{Fall 2024}

%% --> Logos de las guias
\logouno[4.5cm]{../Images/swc_logo}
\definecolor{colordef}{cmyk}{0.81,0.62,0.00,0.22}

%% -- Solucion para alumnos
% \AtBeginEnvironment{proof}{\color{white}}

% https://www.sfu.ca/math-coursenotes/Math%20157%20Course%20Notes/sec_Hopital.html

\begin{document}

\encabezado

\section*{L’Hôpital's Rule}
\begin{mdframed}
    \center Learning Objectives \\
    \begin{itemize}
        \item Recognize limits that produce indeterminate forms.
        \item Apply L'Hôpital's Rule to evaluate these limits.
    \end{itemize}
\end{mdframed}
We begin this section by turning our attention to limits. In Calculus I, there are two main approaches.
\begin{enumerate}
    \item Direct evaluation, e.g.,
    $$
    \lim_{x \to 0} \cos(x) = 1 ~ .
    $$
    \item Algebraic manipulation, e.g.,
    \begin{enumerate}
        \item 
        $$
        \lim_{x \to 3} \frac{x^{2} - 9}{x - 3} \stackrel{\frac{0}{0}}{=} \lim_{x \to 3} \frac{1}{x + 3} = 6,
        $$
        \item
        $$
        \lim_{x \to 0^{+}} \frac{\sqrt{x + 1} - 1}{x^{2}} \stackrel{\frac{0}{0}}{=} \lim_{x \to 0^{+}} \frac{1}{x(\sqrt{x + 1} + 1)} \stackrel{\frac{1}{0^{+}}}{=} \infty,
        $$
        \item 
        $$
        \lim_{x \to \infty} \frac{1-x}{2x} \stackrel{\frac{-\infty}{\infty}}{=} \lim_{x \to \infty} \frac{\frac{1}{x}-1}{2} =  -\frac{1}{2} ~ .
        $$
    \end{enumerate}
\end{enumerate}

Let's now tackle problems where it is not possible to use algebraic tricks. Consider the following limits
$$
    \lim_{x \to 0} \frac{\sin(x)}{x} ~ \text{or} ~ \lim_{x \to \infty} \frac{e^{x}}{x} ~ .
$$

\begin{defi}[\textbf{L’Hôpital's Rule}]
    Suppose $f$ and $g$ are differentiable on an open interval $I$ containing $a$ with $g'(x) \neq 0$ on $I$ when $x \neq a$.

    If $\lim_{x \to a} f(x) = \lim_{x \to a} g(x) = 0$, then
    $$
        \lim_{x \to a} \frac{f(x)}{g(x)} =         \lim_{x \to a} \frac{f'(x)}{g'(x)},
    $$
    provided the limit on the right exists (or is $\pm \infty$). The rule also applies if $x \to a$ is replaced with $x \to \pm \infty, x \to a^{+}, x \to a^{-}$.

    If $\lim_{x \to a} f(x) = \pm \infty$ and $\lim_{x \to a} g(x) = \pm \infty$, then
    $$
        \lim_{x \to a} \frac{f(x)}{g(x)} =         \lim_{x \to a} \frac{f'(x)}{g'(x)},
    $$
    provided the limit on the right exists (or is $\pm \infty$). The rule also applies if $x \to a$ is replaced with $x \to \pm \infty, x \to a^{+}, x \to a^{-}$.
\end{defi}

\begin{ejer}
    Evaluate the limit
    $$
        \lim_{x \to \infty} \frac{3x^{2} + 6x - 1}{5 - 2x^{2}}.
    $$
\end{ejer}
\begin{proof}[Solution]
    We have
    $$
        \lim_{x \to \infty} \frac{3x^{2} + 6x - 1}{5 - 2x^{2}} = \frac{\infty}{\infty}
    $$
    which allows us to invoke L’Hôpital's Rule. Thus,
    \begin{align*}
        \lim_{x \to \infty} \frac{3x^{2} + 6x - 1}{5 - 2x^{2}} 
        & \Heq \lim_{x \to \infty} -\frac{6}{4} \frac{(x + 1)}{x} \\
        & = -\frac{6}{4} \lim_{x \to \infty} \frac{(x + 1)}{x} \\
        & \Heq -\frac{6}{4} \lim_{x \to \infty} \frac{1}{1} \\
        & = -\frac{3}{2} .
    \end{align*}
\end{proof}

\begin{ejer}
    Evaluate the limit
    $$
        \lim_{x \to 0} \frac{4x}{\mathrm{e}^{2x} - 1}.
    $$
\end{ejer}
\begin{proof}[Solution]
    We have
    $$
    \lim_{x \to 0} \frac{4x}{\mathrm{e}^{2x} - 1} = \frac{0}{0}
    $$
    which allows us to invoke L’Hôpital's Rule. Thus,
    \begin{align*}
        \lim_{x \to 0} \frac{4x}{\mathrm{e}^{2x} - 1}
        & \Heq \lim_{x \to 0} \frac{4}{2\mathrm{e}^{2x}} \\
        & = \frac{4}{2} \\
        & = 2 ~ .
    \end{align*}
\end{proof}

\begin{ejer}
    Evaluate the limit
    $$
        \lim_{x \to 0} \frac{\sin(2x)}{5x}.
    $$
\end{ejer}
\begin{proof}[Solution]
    We have
    $$
    \lim_{x \to 0} \frac{\sin(2x)}{5x} = \frac{0}{0}
    $$
    which allows us to invoke L’Hôpital's Rule. Thus,
    \begin{align*}
        \lim_{x \to 0} \frac{\sin(2x)}{5x}
        & \Heq \lim_{x \to 0} \frac{2\cos(2x)}{5} \\
        & = \frac{2}{5} ~ .
    \end{align*}
\end{proof}

\begin{ejer}
    Evaluate the limit
    $$
        \lim_{x \to \infty} x \sin \left( \frac{1}{x} \right).
    $$
\end{ejer}
\begin{proof}[Solution]
    We have
    $$
    \lim_{x \to \infty} x \sin \left( \frac{1}{x} \right) = \infty \cdot 0
    $$
    which means we need to rewrite the expression before applying the limit. Since $ x = \frac{1}{\frac{1}{x}}$, we get
    \begin{align*}
        \lim_{x \to \infty} x \sin \left( \frac{1}{x} \right)
        & = \lim_{x \to \infty} \frac{\sin \left( \frac{1}{x} \right) }{\frac{1}{x}} \\
        & = \lim_{x \to \infty} \frac{\sin \left( \frac{1}{x} \right) }{\frac{1}{x}} \\
        & \Heq \lim_{x \to \infty} \frac{ \frac{d}{dx} \sin \left( \frac{1}{x} \right) }{ \frac{d}{dx} \frac{1}{x}} \\
        & = \lim_{x \to \infty} \frac{ \cos \left( \frac{1}{x} \right) \frac{d}{dx} \frac{1}{x} }{ \frac{d}{dx} \frac{1}{x}} \\
        & = \lim_{x \to \infty}\cos \left( \frac{1}{x} \right) \\
        & = \cos(0) \\
        & = 1 ~ .
    \end{align*}
\end{proof}

\begin{ejer}
    Evaluate the limit
    $$
        \lim_{x \to 0^{+}} x \ln(x).
    $$
\end{ejer}
\begin{proof}[Solution]
    We have
    $$
        \lim_{x \to 0^{+}} x \ln(x) = 0 \cdot (-\infty)
    $$
    which means we need to rewrite the expression before applying the limit. Since $ x = \frac{1}{\frac{1}{x}}$, we get
    \begin{align*}
        \lim_{x \to 0^{+}} x \ln(x)
        & = \lim_{x \to 0^{+}} \frac{\ln(x)}{\frac{1}{x}} \\
        & \Heq \lim_{x \to 0^{+}} \frac{\frac{d}{dx} \ln(x)}{\frac{d}{dx} \frac{1}{x}} \\
        & = \lim_{x \to 0^{+}} \frac{\frac{1}{x}}{-\frac{1}{x^2}} \\
        & = \lim_{x \to 0^{+}} -x \\
        & = 0 ~ .
    \end{align*}
\end{proof}

\begin{ejer}
    Evaluate the limit
    $$
        \lim_{x \to -\infty} x^{2} \mathrm{e}^{x}.
    $$
\end{ejer}
\begin{proof}[Solution]
    We have
    $$
    \lim_{x \to -\infty} x^{2} \mathrm{e}^{x} = - \infty \cdot 0
    $$
    which means we need to rewrite the expression before applying the limit. Note that
    \begin{align*}
        \lim_{x \to -\infty} x^{2} \mathrm{e}^{x}
        & = \lim_{x \to -\infty} \frac{x^{2}}{\mathrm{e}^{-x}} \\
        & \Heq \lim_{x \to -\infty} \frac{2x}{-\mathrm{e}^{-x}} \\
        & \Heq \lim_{x \to -\infty} \frac{2}{\mathrm{e}^{-x}} \\
        & = 0 ~ .
    \end{align*}
\end{proof}

\begin{ejer}
    Evaluate the limit
    $$
        \lim_{x \to 3} \frac{\int_{3}^{x} \sqrt{10 - t^2} dt}{x^{2} - 9}.
    $$
\end{ejer}
\begin{proof}[Solution]
    Direct evaluation yields
    $$
    \lim_{x \to 3} \frac{\int_{3}^{x} \sqrt{10 - t^2} dt}{x^{2} - 9} 
    =  \frac{\int_{3}^{3} \sqrt{10 - t^2} dt}{3^{2} - 9} 
    = \frac{0}{0}
    $$
    which means we need to rewrite the expression before applying the limit. We have
    \begin{align*}
        \lim_{x \to 3} \frac{\int_{3}^{x} \sqrt{10 - t^2} dt}{x^{2} - 9} 
        & \Heq \lim_{x \to 3} \frac{\frac{d}{dx} \int_{3}^{x} \sqrt{10 - t^2} dt}{\frac{d}{dx} (x^{2} - 9)} \\
        & = \lim_{x \to 3} \frac{\sqrt{10 - x^2}}{2x} \\
        & = \frac{1}{6} ~ .
    \end{align*}
\end{proof}

\begin{ejer}
    Evaluate the limit
    $$
        \lim_{x \to 1^{+}} \left( \left( \frac{1}{\ln(x)} \right) - \left( \frac{1}{x - 1} \right) \right) ~ .
    $$
\end{ejer}
\begin{proof}[Solution]
    Direct evaluation yields
    $$
    \lim_{x \to 1^{+}} \left( \left( \frac{1}{\ln(x)} \right) - \left( \frac{1}{x - 1} \right) \right)
    = \infty - \infty
    $$
    which means we need to rewrite the expression before applying the limit. We have
    \begin{align*}
        \lim_{x \to 1^{+}} \left( \left( \frac{1}{\ln(x)} \right) - \left( \frac{1}{x - 1} \right) \right) 
        & = \lim_{x \to 1^{+}} \left( \frac{x - 1 - \ln(x)}{\ln(x)(x - 1)} \right) \\
        & \Heq \lim_{x \to 1^{+}} \left( \frac{ 1 - \frac{1}{x}}{\frac{1}{x} (x - 1) + \ln(x)} \right) \cdot \left( \frac{x}{x} \right)\\
        & = \lim_{x \to 1^{+}} \left( \frac{(x - 1)}{(x - 1) + x \ln(x)} \right) \cdot \left( \frac{\frac{1}{(x-1)}}{\frac{1}{(x-1)}} \right)\\
        & = \lim_{x \to 1^{+}} \left( \frac{1}{1 + \frac{x \ln(x)}{x - 1}} \right) \\
        & = \frac{1}{1 + \displaystyle\lim_{x \to 1^{+}} \frac{x \ln(x)}{x - 1}} \\
        & = \frac{1}{1 + \displaystyle\lim_{x \to 1^{+}} x \cdot \displaystyle\lim_{x \to 1^{+}} \frac{\ln(x)}{x - 1}} ~ .
    \end{align*}
    Notice that
    \begin{align*}
        \lim_{x \to 1^{+}} \frac{\ln(x)}{x - 1}
        & \Heq \lim_{x \to 1^{+}} \frac{\frac{1}{x}}{1} \\ 
        & = 1 ~ .
    \end{align*}
    Therefore,
    $$
        \lim_{x \to 1^{+}} \left( \left( \frac{1}{\ln(x)} \right) - \left( \frac{1}{x - 1} \right) \right) = \frac{1}{2} ~ .
    $$
\end{proof}

\begin{ejer}
    Evaluate the limit
    $$
        \lim_{x \to \infty} \left(\mathrm{e}^{x} - x \right).
    $$
\end{ejer}
\begin{proof}[Solution]
    Direct evaluation yields
    $$
    \lim_{x \to \infty} \left(\mathrm{e}^{x} - x \right)
    = \infty - \infty
    $$
    which means we need to rewrite the expression before applying the limit. We have
    \begin{align*}
        \lim_{x \to \infty} \left(\mathrm{e}^{x} - x \right)
        & = \lim_{x \to \infty} \mathrm{e}^{x} \left(1 - \mathrm{e}^{-x} x \right) \\
        & = \lim_{x \to \infty} \mathrm{e}^{x} \left(1 - \frac{x}{\mathrm{e}^{x}} \right) \\
        & = \lim_{x \to \infty} e^{x} \cdot \lim_{x \to \infty} \left(1 - \frac{x}{\mathrm{e}^{x}} \right) ~ .
    \end{align*}
    Note
    $$
        \lim_{x \to \infty} \left(1 - \frac{x}{\mathrm{e}^{x}} \right) \Heq \lim_{x \to \infty} \left(1 - \frac{1}{\mathrm{e}^{x}} \right) = 1
    $$
    and thus
    $$
        \lim_{x \to \infty} \left(\mathrm{e}^{x} - x \right) = \infty ~ .
    $$
\end{proof}

\begin{ejer}
    Evaluate the limit
    $$
        \lim_{x \to \infty} \left( 1 + \frac{1}{x} \right)^{x}.
    $$
\end{ejer}
\begin{proof}[Solution]
    Let 
    $$
        y = \left( 1 + \frac{1}{x} \right)^{x}
    $$
    so that $\ln(y) = x \ln\left(1 + \frac{1}{x}\right)$. Now
    \begin{align*}
        \lim_{x \to \infty} \ln(y) 
        & = \lim_{x \to \infty} x \ln\left(1 + \frac{1}{x}\right) \\
        & = \lim_{x \to \infty} \frac{\ln\left(1 + \frac{1}{x}\right)}{\frac{1}{x}} \\ 
        & \Heq \lim_{x \to \infty} \frac{\frac{d}{dx} \ln\left(1 + \frac{1}{x} \right)}{\frac{d}{dx} \frac{1}{x}} \\ 
        & = \lim_{x \to \infty} \frac{ (1 + \frac{1}{x})^{-1} \frac{d}{dx} \left(1 + \frac{1}{x} \right)}{\frac{d}{dx} \frac{1}{x}} \\
        & = \lim_{x \to \infty} \frac{1}{1 + \frac{1}{x}} \\
        & = 1 ~ .
    \end{align*}
    It follows that
    \begin{align*}
        \lim_{x \to \infty} \ln(y) & = 1 \\
        \implies \mathrm{e}^{\left( \displaystyle \lim_{x \to \infty} \ln(y) \right)} & = \mathrm{e}^{1} \\
        \implies \lim_{x \to \infty} \mathrm{e}^{\ln(y)} & = \mathrm{e} \\
        \implies \lim_{x \to \infty} y & = \mathrm{e}
    \end{align*}
    $\therefore$
    $$
    \lim_{x \to \infty} \left( 1 + \frac{1}{x} \right)^{x} = \mathrm{e} ~ .
    $$
\end{proof}

\begin{ejer}
    Evaluate the limit
    $$
        \lim_{x \to 0^{+}} \left( x \right)^{2x}.
    $$
\end{ejer}
\begin{proof}[Solution]
    Direct evaluation yields
    $$
        \lim_{x \to 0^{+}} \left( x \right)^{2x}
        = 0^{0}
    $$
    which means we need to rewrite the expression before applying the limit. Let $y = x^{2x}$ so that $\ln(y) = 2x \ln(x)$. Thus,
    \begin{align*}
        \lim_{x \to 0^{+}} \ln(y)
        & = \lim_{x \to 0^{+}} 2x \ln(x) \\
        & = \lim_{x \to 0^{+}} 2 \frac{\ln(x)}{\frac{1}{x}} \\
        & \Heq \lim_{x \to 0^{+}} 2 \frac{\frac{d}{dx} \ln(x)}{\frac{d}{dx} \frac{1}{x}} \\
        & = \lim_{x \to 0^{+}} 2 \frac{\frac{1}{x}}{-\frac{1}{x^2}} \\
        & = \lim_{x \to 0^{+}} -2 \frac{1}{\frac{1}{x}} \\ 
        & = \lim_{x \to 0^{+}} -2 x \\
        & = 0 ~ .
    \end{align*}
    It follows that
    \begin{align*}
        \lim_{x \to 0^{+}} \ln(y) & = 0 \\
        \implies \mathrm{e}^{\left( \displaystyle \lim_{x \to 0^{+}} \ln(y) \right)} & = \mathrm{e}^{0} \\
        \implies \lim_{x \to 0^{+}} \mathrm{e}^{\ln(y)} & = 1 \\
        \implies \lim_{x \to 0^{+}} y & = 1 ~ .
    \end{align*}
\end{proof}

\end{document}